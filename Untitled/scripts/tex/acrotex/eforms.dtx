% \iffalse
% makeindex -s gglo.ist -o eforms.gls eforms.glo
% makeindex -s gind.ist -o eforms.ind eforms.idx
%<*copyright>
%%%%%%%%%%%%%%%%%%%%%%%%%%%%%%%%%%%%%%%%%%%%%%%%%%%%%%%%%%
%% eForms package,                                      %%
%% Copyright (C) 2002-2017  D. P. Story                 %%
%%   dpstory@uakron.edu                                 %%
%%                                                      %%
%% This program can redistributed and/or modified under %%
%% the terms of the LaTeX Project Public License        %%
%% Distributed from CTAN archives in directory          %%
%% macros/latex/base/lppl.txt; either version 1 of the  %%
%% License, or (at your option) any later version.      %%
%%%%%%%%%%%%%%%%%%%%%%%%%%%%%%%%%%%%%%%%%%%%%%%%%%%%%%%%%%
%</copyright>
%<package>\NeedsTeXFormat{LaTeX2e}
%<package>\ProvidesPackage{eforms}
%<package> [2017/09/04 v2.9n Provides general eforms support (dps)]
%<*driver>
\documentclass{ltxdoc}
\usepackage[colorlinks,hyperindex=false]{hyperref}[2012/10/12] % to support calculate for pdfyrc
\pdfstringdefDisableCommands{\let\\\textbackslash}
\OnlyDescription
\EnableCrossrefs
\CodelineIndex
\RecordChanges
\def\ltag{<}\def\rtag{>}
\let\app\textsf\let\pkg\textsf
\InputIfFileExists{aebdocfmt.def}{\PackageInfo{eforms}{Inputting aebdocfmt.def}}
    {\def\IndexOpt{\DescribeMacro}\def\IndexKey{\DescribeMacro} \let\setupFullwidth\relax
     \PackageInfo{eforms}{aebdocfmt.def cannot be found}}
\begin{document}
  \GetFileInfo{eforms.sty}
  \title{eForm: PDF Form support for \LaTeX}
  \author{D. P. Story\\
    Email: \texttt{dpstory@uakron.edu}}
  \date{processed \today}
  \maketitle
  \tableofcontents
  \let\Email\texttt
  \DocInput{eforms.dtx}
\IfFileExists{\jobname.ind}{\newpage\setupFullwidth\PrintIndex}{\paragraph*{Index} The index goes here.\\Execute
    \texttt{makeindex -s gind.ist -o eforms.ind eforms.idx} on the command line and recompile
    \texttt{eforms.dtx}.}
\IfFileExists{\jobname.gls}{\PrintChanges}{\paragraph*{Change History} The list of changes goes here.\\Execute
    \texttt{makeindex -s gglo.ist -o eforms.gls eforms.glo} on the command line and recompile
    \texttt{eforms.dtx}.}
\end{document}
%</driver>
% \fi
%
% \MakeShortVerb{|}
% \InputIfFileExists{aebdonotindex.def}{\PackageInfo{eforms}{Inputting aebdonotindex.def}}
%    {\PackageInfo{eforms}{aebdonotindex.def cannot be found}}
%
%    \begin{macrocode}
%<*package>
%    \end{macrocode}
%
%    \section{Introduction}
%
% This {\LaTeX} code was originally part of the \textsf{exerquiz} package.
% I decided it would be useful to others if I separated the two and make
% the forms part of \textsf{exerquiz} into a separate package.  The
% \textsf{eforms} Package is now a part of the \textsf{{Acro\TeX} eDucation
% Bundle} and is called by \textsf{exerquiz}, but it is  now a stand alone
% package for others who may want to use PDF form fields and JavaScript
% interactivity.
%
%    \section{Package Options}
%
% The package options consist mostly of driver options.
%
%    \subsection{Driver Options}
%
%    The \textsf{web} package passes these driver options to \textsf{exerquiz}.
%    These options are needed is \textsf{exerquiz} is used without
%    \textsf{web}; in this case, the options below must explicitly included.
%    Set the driver dependent code for the |quiz| environments.
%
% \changes{v2.5o}{2012/06/18}{Added required package \string\textsf{ifpdf}}
% \changes{v2.9m}{2017/09/03}{Added the ifluatex package}
%    \begin{macrocode}
\RequirePackage{ifpdf}[2006/02/20]
\RequirePackage{ifxetex}[2006/08/21]
\RequirePackage{ifluatex}
%    \end{macrocode}
% \changes{v2.6c}{2014/02/18}{Added the \string\textsf{calc} package as required package.}
%    \begin{macrocode}
\RequirePackage{calc}
%    \end{macrocode}
% Set the driver for \texttt{dvipsone}\IndexOpt{dvipsone}
%    \begin{macrocode}
\let\ef@driver\@empty
\DeclareOption{dvipsone}{\def\eq@drivernum{0}%
    \def\eq@drivername{0}\def\ef@driver{dvipsone}%
    \PassOptionsToPackage{dvipsone}{insdljs}%
    \PassOptionsToPackage{dvipsone}{hyperref}%
}
\def\eq@drivername{2}
%    \end{macrocode}
% Set the driver for \texttt{dvips}\IndexOpt{dvips}
%    \begin{macrocode}
\DeclareOption{dvips}{\def\eq@drivernum{0}%
    \def\eq@drivername{1}\def\ef@driver{dvips}%
    \PassOptionsToPackage{dvips}{insdljs}%
    \PassOptionsToPackage{dvips}{hyperref}%
}
%    \end{macrocode}
% Set the driver for \texttt{pdftex}\IndexOpt{pdftex}
%    \begin{macrocode}
\DeclareOption{pdftex}{%
    \def\eq@drivernum{1}\def\eq@driver{pdftex}%
    \def\eq@drivercode{epdftex.def}\def\ef@driver{pdftex}%
%    \end{macrocode}
% \changes{v2.9b}{2016/07/22}{Do not pass pdftex driver to insdljs or hyperref}
%    \begin{macrocode}
%    \PassOptionsToPackage{pdftex}{insdljs}
%    \PassOptionsToPackage{pdftex}{hyperref}
}
%    \end{macrocode}
% Added \opt{luatex} option
% \changes{v2.9m}{2017/09/03}{Add luatex option}
%    \begin{macrocode}
\DeclareOption{luatex}{%
    \def\eq@drivernum{1}\def\eq@driver{luatex}%
    \def\eq@drivercode{epdftex.def}\def\ef@driver{luatex}%
}
%    \end{macrocode}%
% Set the drivers for \texttt{dvipdfm}\IndexOpt{dvipdfm}. \texttt{dvipdfmx}\IndexOpt{dvipdfmx},
% and \texttt{xetex}\IndexOpt{xetex}.
%    \begin{macrocode}
\DeclareOption{dvipdfm}{%
    \def\eq@drivernum{2}\def\eq@driver{dvipdfm}%
    \def\eq@drivercode{edvipdfm.def}\def\ef@driver{dvipdfm}%
    \PassOptionsToPackage{dvipdfm}{insdljs}
    \PassOptionsToPackage{dvipdfm}{hyperref}
}
\DeclareOption{dvipdfmx}{%
    \def\eq@drivernum{2}\def\eq@driver{dvipdfmx}%
    \def\eq@drivercode{edvipdfm.def}\def\ef@driver{dvipdfmx}%
    \PassOptionsToPackage{dvipdfmx}{insdljs}
    \PassOptionsToPackage{dvipdfmx}{hyperref}
}
\DeclareOption{xetex}{%
    \def\eq@drivernum{2}\def\eq@driver{xetex}%
    \def\eq@drivercode{edvipdfm.def}\def\ef@driver{xetex}%
%    \end{macrocode}
% \changes{v2.9b}{2016/07/22}{Do not pass xetex driver to insdljs or hyperref}
%    \begin{macrocode}
%    \PassOptionsToPackage{xetex}{insdljs}
%    \PassOptionsToPackage{xetex}{hyperref}
}
%    \end{macrocode}
%    \IndexOpt{textures}
% This option, and testing are due to Ross Moore  3/6/02
%    \begin{macrocode}
\DeclareOption{textures}{%
    \def\eq@drivernum{3}\def\eq@driver{textures}%
    \def\eq@drivercode{epdfmark.def}\def\ef@driver{textures}%
    \PassOptionsToPackage{textures}{insdljs}%
    \PassOptionsToPackage{textures}{hyperref}
}
%    \end{macrocode}
% If no driver is passed, assume it is
% \textsf{dvipsone} or \textsf{dvips}---\textsf{hyperref} defines the specials.
% Default driver dvipsone/dvips
%    \begin{macrocode}
% \def\eq@drivernum{0}
\def\eq@driver{dvipsone/dvips}
\def\eq@drivercode{epdfmark.def}
%    \end{macrocode}
%
% \subsection{Other Options}
%
% The \texttt{preview}\IndexOpt{preview} option displays the bounding box of each form field with a frame box.
% Useful for laying out field with a dvi previewer.
% \changes{v2.5n}{2012/04/30}{Changed \string\cs{ifpreview} to a conditional
% definition; this allows the \string\textsf{spdef} package to control the state
% of \cs{ifpreview}.}
%    \begin{macrocode}
\DeclareOption{preview}{\previewtrue}
\DeclareOption*{\PassOptionsToPackage{\CurrentOption}{insdljs}}
\@ifundefined{ifpreview}{\newif\ifpreview \previewfalse}{}
%    \end{macrocode}
%    (2017/01/01) Added two convenience commands.
%    \changes{v2.9f}{2017/01/01}{Added \string\cs{previewOn} and \string\cs{previewOff}}
%    \begin{macrocode}
\providecommand{\previewOn}{\previewtrue}
\providecommand{\previewOff}{\previewfalse}
%    \end{macrocode}
%    Use\IndexOpt{useui} the \textsf{xkeyval} package to specify the options for the links and
% forms, key-values are enclosed in the \cs{ui} command inside the option list.
% \changes{v2.0}{2008/03/14}
% {
%   Added the \string\texttt{useui} option, which inputs \string\textsf{xkeyval} package,
%   and defines a user-friendly interface to the option arguments.
% }
%    \begin{macrocode}
\DeclareOption{useui}{\AtEndOfPackage{\ef@InputUIeForms}}
\def\ef@InputUIeForms{\InputIfFileExists{uieforms.def}%
    {\PackageInfo{eforms}{Inputting code for useui option}}%
    {\PackageInfo{eforms}{Cannot find uieforms.def.\MessageBreak
        Reinstall or refresh your file name database.}}}
%    \end{macrocode}
%    The \texttt{setcorder} option\IndexOpt{setcorder} is used to set the calculation order
%    in a forms document. Normally, the calculation order is the order the fields are created.
%    \begin{macrocode}
\DeclareOption{setcorder}{\def\inputCalcOrderJS{%
    \InputIfFileExists{setcorder.def}%
    {\PackageInfo{eforms}{Inputting code for setcorder option}}%
    {\PackageWarning{eforms}{Cannot find setcorder.def.\MessageBreak
        Reinstall or refresh your file name database.}}}}
\let\inputCalcOrderJS\relax
%    \end{macrocode}
% \changes{v2.5o}{2012/06/18}{Added automatic test for \string\textsf{pdftex}.}
% \changes{v2.9m}{2017/09/03}{Added a check for \string\textsf{lualtex}}
%    \begin{macrocode}
\@ifpackageloaded{web}{%
    \ExecuteOptions{\eq@driver@name}%
}{%
    \@ifpackageloaded{exerquiz}{%
        \ExecuteOptions{\eq@driver}%
    }{%
        \ifluatex\ExecuteOptions{luatex}\else
        \ifpdf\ExecuteOptions{pdftex}\else
        \ifxetex\ExecuteOptions{xetex}\else
        \@ifundefined{l@tex@@@@driver}{\ExecuteOptions{dvips}}
            {\ExecuteOptions{dvipsone}}\fi\fi\fi
    }%
}
%    \end{macrocode}
%    \begin{macrocode}
\ProcessOptions
\ifx\ef@driver\@empty
    \PackageError{eforms}%
        {You have not specified dvips, dvipsone, pdftex,\MessageBreak
        dvipdfm, dvipdfmx, or xetex in the option list of\MessageBreak
        the eforms package}
        {Place one of the drivers dvips, dvipsone, pdftex,  dvipdfm,
        dvipdfmx, or xetex
        \MessageBreak in the option list of the eforms package.}%
}{}
\fi
\newlength\eflength
\@ifundefined{ifpdfmarkup}{\newif\ifpdfmarkup}{}\pdfmarkupfalse
\ifpdf\else\ifxetex\else\pdfmarkuptrue\fi\fi
\RequirePackage{hyperref}
%    \end{macrocode}
%    Changed \textsf{hyperref} definitions/parameters to add \texttt{1bp} rather than \texttt{1pt} around a form field
%    or link.
%    \changes{v2.9e}{2016/12/22}{Changed hyperref to add 1bp rather than 1pt}
%    \begin{macrocode}
\ifxetex\else\ifpdf\pdflinkmargin1bp\relax\else
    \g@addto@macro\Hy@FirstPageHook{%
    \headerps@out{/HyperBorder {1.00375 PDFToDvips} def}}
\fi\fi
%    \end{macrocode}
%    Changed the requirement for \textsf{insdljs}
% \changes{v2.9a}{2016/06/09}{Change in \string\textsf{insdljs} package}
% \changes{v2.9g}{2017/01/03}{Change in \string\textsf{insdljs} package}
%    \begin{macrocode}
\RequirePackage{insdljs}[2017/01/03] % incl conv-xkv
%    \end{macrocode}
% \changes{v1.0a}{2006/10/03}
% {
%   If \string\textsf{exerquiz} is not loaded, we do an automatic begin and end of Form.
%   Also, if \string\textsf{exerquiz} is loaded, then we use the driver declared in
%   \string\textsf{exerquiz}; otherwise, we set the default to 0
%   (\string\texttt{dvipsone}/\string\texttt{dvips}).
% }
% If exerquiz is not loaded, we insert |\begin{Form}| and |\end{Form}|,
% and if undefined, we set the default driver.
%    \begin{macrocode}
\@ifpackageloaded{exerquiz}{}{%
    \AtBeginDocument{\Form}
    \AtEndDocument{\csname endForm\endcsname}
}
\@ifpackageloaded{aeb_pro}{}{%
    \newcommand{\taggedPDF}{%
        \ifnum\eq@drivernum=0\relax
        \literalps@out{[{Catalog} <<%
            /MarkInfo<</Marked true>>%
        >> /PUT pdfmark}\fi}%
}
\let\ef@YES=y \let\ef@NO=n
\let\ef@One=1 \let\ef@Zero=0
\ifnum\eq@drivername<2
\let\to@usepdfmark\ef@One
\RequirePackage[structure]{taborder}\else
\let\to@usepdfmark\ef@Zero
\RequirePackage{taborder}
\fi
%    \end{macrocode}
%    \begin{macrocode}
%</package>
%    \end{macrocode}
% When the \texttt{preview} option is used, draw a frame box
% around the \textit{inner} bounding rectangle.
%    \begin{macrocode}
%<*package>
\@ifundefined{eq@tmpbox}{\newsavebox{\eq@tmpbox}}{}
\@ifundefined{eq@tmpdima}{\newdimen\eq@tmpdima}{}
\@ifundefined{eq@tmpdimb}{\newdimen\eq@tmpdimb}{}
\newlength\ef@dimena
\newtoks\ef@scratchtoks
%    \end{macrocode}
%    (2016/12/22) Added switches \cs{ifmakeXasPD} and \cs{ifmakePDasX}, if true, the form fields created
%    by \app{xetex} (\app{pdflatex/Distiller})
%    are adjusted in dimension to conform to the fields produced by \app{pdftex}/\app{Distiller} (\app{xetex}).
%    Four convenience commands are defined, \DescribeMacro{\makeXasPDOn}\cs{makeXasPDOn},
%    \DescribeMacro{\makeXasPDOff}\cs{makeXasPDOff}, \DescribeMacro{\makePDasXDOn}\cs{makePDasXDOn}, and
%    \DescribeMacro{\makePDasXDOff}\cs{makePDasXDOff} to set the switch to true and false, respectively.
%    \changes{v2.9d}{2016/12/22}{Added switches \string\cs{ifmakeXasPD} and \string\cs{ifmakePDasX}}
%    \begin{macrocode}
\newif\ifmakeXasPD \makeXasPDtrue % 12/22
\newif\ifmakePDasX \makePDasXfalse
\def\makeXasPDOn{\makeXasPDtrue\makePDasXfalse}
\def\makeXasPDOff{\makeXasPDfalse}
\def\makePDasXOn{\makePDasXtrue\makeXasPDfalse}
\def\makePDasXOff{\makePDasXfalse}
%    \end{macrocode}
% \changes{v2.8a}{2015/06/06}{Added \string\cs{previewColor}}
% \DescribeMacro\previewColor sets the color of the preview bounding rectangle. The
% default is black. Used mostly by the \textsf{eqexam} package with the \texttt{online}
% option.
%    \begin{macrocode}
\providecommand\previewColor{black}
\def\ef@Bbox#1#2{%
    \hbox{\ifpreview\setlength\fboxrule{0.4pt}\setlength\fboxsep{0pt}%
    \@tempdima=#1\advance\@tempdima by-\fboxrule
    \@tempdimb=#2\advance\@tempdimb by-\fboxrule\color{\previewColor}%
    \fbox{\parbox[b][\@tempdimb][c]{\@tempdima}{\hfill\vfill}}\else
    \parbox[b][#2][c]{#1}{\hfill\vfill}\fi}%
}
\let\Bbox\ef@Bbox
%</package>
%    \end{macrocode}
%    \section{eForms Support}
%    \begin{macrocode}
%<*package>
%    \end{macrocode}
% Listed below are the various types of form objects available in PDF.
%\begin{itemize}
% \item \hyperref[button]{Button}
%   \begin{itemize}
%       \item \hyperref[pushbutton]{Push buttons}
%       \item \hyperref[checkbox]{Check boxes}
%       \item \hyperref[radiobutton]{Radio Buttons}
%   \end{itemize}
% \item \hyperref[textfield]{Text Fields}
% \item \hyperref[choice]{Choice Fields}
%   \begin{itemize}
%       \item \hyperref[listbox]{list box}
%       \item \hyperref[combobox]{combo box}
%   \end{itemize}
%\item \hyperref[sigfield]{Signature Fields}
%\end{itemize}
% \normalcolor
%
% \subsection{Process Key-Value Pairs: Main Macro}\label{procArgs}
%
% The following macro, \cs{processAppArgs},  is  due in part to
% Dan Luecking. He proposed a very nice modification of my original
% macros.
%
% The macro \cs{processAppArgs} takes an \textit{even} number of arguments; it
% picks off two at a time, processes them, then picks off two more. The macro is meant to
% process the optional arguments of a form field.
%
% All legal arguments (see \nameref{eformvariables} for a detailed
% listing) are of the form |\<name>{arg}|.
% The macro takes two tokens at a time and constructs a macro
% |\@eq<name>{arg}|. Each of the macros |\@eq<name>| must be
% defined. Such a macro defines another macro as follows
% |\def\eq@<name>{arg}|. For example if the user enters the token
% pair  |\RC{Users}|, \cmd{\processAppArgs} will construct
% |\@eqRC|, with argument \verb+{arg}+, this macro will be
% executed, which expands to \verb+\def\eq@RC{arg}+.  The macro
% \cmd{\eq@RC} may then be used within the construction of the
% widget object.
%
% The macro \cs{processAppArgs} also does the addition operation for
% \texttt{/F} and \texttt{/Ff} keys.
%    \begin{macrocode}
\let\ef@passedArgs\@empty
\def\processAppArgs#1#2{%
    \ifx\end#1% if #1=\end, #2=\@nil.
        \let\ef@next\relax
    \else
%    \end{macrocode}
% If a token has a value of \cs{@empty} then it has been protected. It is skipped
% and there is no user redefinition of that form field attribute allowed. Normally,
% this is done for \cs{A} and \cs{AA} to prevent overwriting critical functionality.
%    \begin{macrocode}
        \ifx#1\@empty
            \def\ef@next{\processAppArgs}%
        \else
%    \end{macrocode}
% This is the user interface to the new optional argument of links and forms. If
% the key is \cs{ui}, we pass its argument to |\setkeys{eforms}{#2}| to process
% the key values of \textsf{xkeyval} style. If one of the keys is \texttt{annotflags} or
% \texttt{fieldflags}, we pass those back to this stream to be analyzed the special cases
% that follow for |\F| and |\Ff|.
%    \begin{macrocode}
            \@getCmdName{\ui}\edef\arg@ui{\@CmdName}%
            \@getCmdName{#1}%
            \ifx\arg@ui\@CmdName
                \@ifundefined{@equi}{\PackageError{eforms}%
                {The user interface '\string\ui' is not defined!%
                \MessageBreak Use the useui option of eforms
                and try again}{I said, use the useui option of
                eforms and try again!}}{}%
                \def\ef@next{\setkeys{eforms}{#2}%
                \processAppArgs\presets{\ef@passedArgs}}%
            \else
%    \end{macrocode}
% If current key is |\Ff|, we add its value to the current value of |\Ff|.
% We basically are `or-ing' the new value with the old value in the bit field.
%    \begin{macrocode}
                \@getCmdName{\Ff}\edef\arg@Ff{\@CmdName}%
                \@getCmdName{#1}%
                \ifx\arg@Ff\@CmdName  % if \Ff, let's  add arguments
                    \ifx\eq@Ff\@empty\def\eq@FfValue{0}\else
                        \expandafter\getFfValue\eq@Ff\@nil\fi
                    \@tempcnta=\eq@FfValue
                    \def\eq@arg{#2}%
                    \ifx\eq@arg\@empty\else
                    \advance\@tempcnta by#2\fi
                    \edef\eq@Ff{/Ff \the\@tempcnta}%
                    \def\ef@next{\processAppArgs}%
                \else
%    \end{macrocode}
% If current key is |\F|, we add its value to the current value of |\F|.
% We basically are ```or-ing'' new value with the old value in the bit field.
%    \begin{macrocode}
                    \@getCmdName{\F}\edef\arg@F{\@CmdName}%
                    \@getCmdName{#1}%
                    \ifx\arg@F\@CmdName  % if \Ff, let's  add arguments
                    \ifx\eq@F\@empty\def\eq@FValue{0}\else
                        \expandafter\getFValue\eq@F\@nil\fi
                    \@tempcnta=\eq@FValue
                    \def\eq@arg{#2}%
                    \ifx\eq@arg\@empty\else
                    \advance\@tempcnta by#2\fi
                    \edef\eq@F{/F \the\@tempcnta}%
                    \def\ef@next{\processAppArgs}%
                    \else
%    \end{macrocode}
% If the key we are processing is \cs{presets}, then use \cs{expandafter} to
% expand its argument (it is assumed the argument is a macro), then put it back
% into the input stream.
%    \begin{macrocode}
                        \@getCmdName{\presets}%
                        \edef\arg@presets{\@CmdName}\@getCmdName{#1}%
                        \ifx\arg@presets\@CmdName
                            \def\ef@next{\expandafter\processAppArgs#2}%
                        \else
%    \end{macrocode}
% This is the last, and the most frequent case. We process a common key, one
% that is not |\presets|, |\ui|, |\Ff| or |\F|.
%    \begin{macrocode}
                            \csname @eq%
                            \expandafter\@gobble\string#1\endcsname{#2}%
                            \def\ef@next{\processAppArgs}%
                        \fi
                    \fi
                \fi
            \fi
        \fi
    \fi
    \ef@next
}
%    \end{macrocode}
% Process the field defaults and the `every' changes. Build up the required command
% in a token list, then execute.
%    \begin{macrocode}
\def\@processEvery{\edef\eqtemp{}\toks0={}\@@processEvery}
\def\@@processEvery#1{%
    \ifx#1\end\def\ef@next{\the\toks0 }%
    \else
        \edef\eqtemp{\the\toks0 }%
        \toks0=\expandafter{\eqtemp
            \expandafter\processAppArgs#1\end\@nil}%
        \def\ef@next{\@@processEvery}%
    \fi
    \ef@next
}
%    \end{macrocode}
%    \begin{macrocode}
\newdimen\eqcenterWidget
%    \end{macrocode}
% This macro is used to vertically center text fields and buttons on a
% line. Seems to work well.
% \changes{v2.5h}{2012/11/17}{Introduce the \string\cs{inline} key designed for
% inline form fields.}
%    \begin{macrocode}
\def\centerWidget#1{%
    \ifeq@inlineCenter
%    \end{macrocode}
% Inline form field, do a better job at centering it.
%    \begin{macrocode}
        \eqcenterWidget=#1\relax
        \eqcenterWidget=.5\eqcenterWidget
        \ifnum\eq@textSize=0\relax
        \dimen@=-\eq@textSizeDefault bp\else
        \dimen@=-\eq@textSize bp\fi
        \dimen@=0.9167\dimen@ % 11/12
        \dimen@=.5\dimen@
        \advance\dimen@\eq@W@value bp
        \ifx\eq@S\@empty\else
            \def\eq@S@cmp{B}%
            \ifx\eq@S@value\eq@S@cmp
                \advance\dimen@ by \eq@W@value bp
            \else
                \def\eq@S@cmp{I}%
                \ifx\eq@S@value\eq@S@cmp
                    \advance\dimen@ by \eq@W@value bp
                \else\advance\dimen@ by 1bp
        \fi\fi\fi
        \advance\eqcenterWidget by \dimen@
    \else
        \eqcenterWidget=#1\relax
        \eqcenterWidget=.5\eqcenterWidget
        \advance\eqcenterWidget by -4bp
    \fi
}
%    \end{macrocode}
% \subsection{eForm Variables}\label{eformvariables}
%
% \subsubsection{Key-Value Definitions}
%
% The following definitions are used in various field templates.
% Some convenience macros to help define the button attributes. The default
% values are defined within the button macros themselves. Use these macros
% within the optional argument of buttons and text fields to modify their
% appearance.
%
% You'll notice, for example, the macros listed are not actually defined. For example
% \cmd{\CA} is never actually defined; we define instead \cmd{\@eqCA} and \cmd{\eq@CA}.
% The macros \cmd{\processAppArgs} treats \cmd{\CA} as a token, and prefixes with
% \texttt{@eq} in a clever sort of way. It's done so that these macros cannot be used
% outside the optional macro arguments of the button and text field macros.
%
% \paragraph*{Entries common to all annotation dictionaries:}
% \texttt{F, BS, Border, AP, AS, T, A, AA}.
%
% \medskip\noindent\textsl{Annotation Flag Bit Field:} See \nameref{F} for values.
%    \begin{macrocode}
\def\@eqF#1{\def\eq@arg{#1}\ifx\eq@arg\@empty
    \let\eq@F\@empty\else\def\eq@F{/F #1}\fi}\def\eq@F{}
%    \end{macrocode}
% \DescribeMacro{\BS}The \textbf{Border Style} key, \texttt{BS}: \texttt{W}, \texttt{S}, \texttt{D}  (dictionary, optional)
%    \begin{macrocode}
\def\@eqBS#1{%
    \let\eq@BS=0\relax
    \ifx\eq@W\@empty\else\let\eq@BS=1\fi
    \ifx\eq@S\@empty\else\let\eq@BS=1\fi
    \ifx\eq@D\@empty\else\let\eq@BS=1\fi
    \edef\link@BS{\if\eq@BS1/BS<<\eq@W\eq@S\eq@D>>\fi}%
    \ifx\eq@W\@empty\let\link@BS\@empty\fi
}\def\link@BS{}
%    \end{macrocode}
% \changes{v1.0e}{2008/03/04}
% {
%   Added a \string\cs{presets} key to make it easier to dynamically change options%
% }
%
% \noindent\DescribeMacro{\presets} We define the \texttt{presets} key. The argument of presets is a macro, that is
% expanded and put back into the parsing stream.
%    \begin{macrocode}
\def\@eqpresets#1{#1}%
%    \end{macrocode}
% \DescribeMacro{\W} The width of the boundary line.
%    \begin{macrocode}
\def\@eqW#1{\def\eq@arg{#1}\ifx\eq@arg\@empty
    \let\eq@W\@empty\def\eq@W@value{0}\else
    \def\eq@W@value{#1}\def\eq@W{/W #1}\fi
%    \end{macrocode}
%    (2016/12/22) Add a global value for boundary width, used to adjust the spacing between form fields
%    \changes{v2.9d}{2016/12/22}{Add a global value for boundary width}
%    \begin{macrocode}
    \xdef\g@eq@W@value@bp{\eq@W@value bp}}
\def\eq@W{}\def\eq@W@value{0}
\def\g@eq@W@valu@bp{0bp} % dps 12/22
%    \end{macrocode}
% \DescribeMacro{\S} Line style, values are \texttt{S} (solid), \texttt{D} (dashed),
%      \texttt{B} (beveled), \texttt{I} (inset), \texttt{U} (underlined)
%    \begin{macrocode}
\def\@eqS#1{\def\eq@S@value{#1}\ifx\eq@S@value\@empty
    \let\eq@S\@empty\else
    \def\eq@S{/S/#1}\def\eq@temp{D}%
    \ifx\eq@S@value\eq@temp
        \ifx\eq@D\@empty\def\eq@D{/D [3]}\fi
    \fi\fi}\def\eq@S{}
%    \end{macrocode}
% \DescribeMacro{\D} The dash array.
%    \begin{macrocode}
\def\@eqD#1{\def\eq@arg{#1}\ifx\eq@arg\@empty
    \let\eq@D\@empty\else
    \def\eq@D{/D [#1]}\fi}\def\eq@D{}
%    \end{macrocode}
% \DescribeMacro{\Border} Used with \emph{link annotations}, an array of three numbers and an optional dash array.
%   If all three numbers are 0, no border is drawn
%    \begin{macrocode}
\def\@eqBorder#1{\def\eq@arg{#1}\ifx\eq@arg\@empty
    \let\eq@Border\@empty\else\def\eq@Border{/Border [#1]}\fi}%
\def\eq@Border{/Border [0 0 0]}
%    \end{macrocode}
% \DescribeMacro{\AP} Appearance dictionary, used mostly with check boxes
%    to define the `On' value.
%    \begin{macrocode}
\def\@eqAP#1{\def\eq@arg{#1}\ifx\eq@arg\@empty
    \let\eq@AP\@empty\else\def\eq@AP{/AP<<#1>>}\fi}%
    \let\eq@AP\@empty
%    \end{macrocode}
% In the \texttt{AP} dictionary for checkboxes is the `On' value.
% It is introduced into \texttt{AP} by passing a TeX parameter
% normally, this variable is not used.
%    \begin{macrocode}
    \def\@eqOn#1{\def\eq@On{/#1}}\def\eq@On{/Yes}
%    \end{macrocode}
% \DescribeMacro{\AS} Appearance state, normally used with check boxes and radio buttons when there are
%  more than one appearance. Advanced techniques only.
%    \begin{macrocode}
\def\@eqAS#1{\def\eq@arg{#1}\ifx\eq@arg\@empty
    \let\eq@AS\@empty\else\ifpdfmarkup\def\eq@AS{/AS(#1) cvn }\else
    \def\eq@AS{/AS/#1}\fi\fi}\def\eq@AS{}
%\def\eq@setAS#1{\ifx\annot@type\annot@type@checkbox\@eqAS{#1}\else
%    \ifx\annot@type\annot@type@radio\@eqAS{#1}\fi\fi}
%    \end{macrocode}
% \paragraph*{The A Dictionary.} In the \texttt{A} dictionary for actions.
%\changes{v2.5a}{2009/12/22}
%{%
% Added special commands for processing the optional argument
% for links. We search for keys words in order to set the
% correct link color. \string\cs{ef@preprocessA} was defined to
% execute these search routines, this is introduced
% for in \string\cs{@eqA} (for \string\cs{rPage}). For links and
% buttons, the command \string\cs{ef@preProcDefns} is also
% inserted.
%}
% Added special commands for processing the optional argument
% for links. We search for keys words in order to set the
% correct link color. \cs{ef@preprocessA} was defined to
% execute these search routines, this is introduced
% for in \cs{@eqA} (for \cs{rPage}). For links and
% buttons, the command \cs{ef@preProcDefns} is also
% inserted.
%    \begin{macrocode}
\def\ef@gobbleToendmarker#1\ef@endmarker{}
\let\ef@endmarker\relax
%    \end{macrocode}
% \textsf{eform} definitions of \texttt{true} and \texttt{false}, used in the search
% algorithm.
%    \begin{macrocode}
\def\ef@end{\end}\def\ef@true{true}
%    \end{macrocode}
% This is the internal definition of |rPage|, a command is used
% in the destination array and the \texttt{GoToR} action. When
% we jump to page number, the number must be zero-based, so we
% take the number provided by the author (1-based), descrement
% by one, and re-define |rPage|.
%    \begin{macrocode}
\def\ef@rPage#1{\@tempcnta=#1\relax\advance\@tempcnta-1
    \edef\rPage##1{\the\@tempcnta}}
%    \end{macrocode}
% We search for |\rPage| in the argument of |\eq@A|, get the page
% number and decrement it.
%    \begin{macrocode}
\long\def\ef@searchrPage#1\rPage#2#3\@nil{\def\ef@argii{#2}%
    \ifx\ef@argii\ef@end\else\rPage{#2}\fi}
%    \end{macrocode}
% When the user specifies |\mlLink{true}| in the option list, we branch off to
% \cs{mlhypertext}.
%    \begin{macrocode}
\def\ef@searchmlLink#1\mlLink#2#3\@nil{\def\ef@argii{#2}%
    \ifx\ef@argii\ef@end\let\ef@mlLink=0\else
    \ifx\ef@argii\ef@true\let\ef@mlLink=1\else\let\ef@mlLink=0\fi\fi}
%    \end{macrocode}
% Search for \texttt{/GoToR}, if found, change the link color to |\@filecolor|
%    \begin{macrocode}
\def\ef@searchGoToR#1/GoToR#2\@nil{\def\ef@argii{#2}%
    \ifx\ef@argii\ef@end\else\ifx\ef@linktxtcolor@set0%
    \def\ef@thislinkcolor{\@filecolor}\fi
    \expandafter\ef@gobbleToendmarker\fi}
%    \end{macrocode}
% Search for \texttt{/URI}, if found, change the link color to |\@urlcolor|
%    \begin{macrocode}
\def\ef@searchURI#1/URI#2\@nil{\def\ef@argii{#2}%
    \ifx\ef@argii\ef@end\else\ifx\ef@linktxtcolor@set0%
    \def\ef@thislinkcolor{\@urlcolor}\fi
    \expandafter\ef@gobbleToendmarker\fi}
\def\ef@searchCmdURI#1\URI#2\@nil{\def\ef@argii{#2}%
    \ifx\ef@argii\ef@end\else\ifx\ef@linktxtcolor@set0%
    \def\ef@thislinkcolor{\@urlcolor}\fi
    \expandafter\ef@gobbleToendmarker\fi}
%    \end{macrocode}
% Search for \texttt{/Named}, if found, change the link color to |\@menucolor|
%    \begin{macrocode}
\def\ef@searchNamed#1/Named#2\@nil{\def\ef@argii{#2}%
    \ifx\ef@argii\ef@end\else\ifx\ef@linktxtcolor@set0%
    \def\ef@thislinkcolor{\@menucolor}\fi
    \expandafter\ef@gobbleToendmarker\fi}
\def\ef@searchCmdNamed#1\Named#2\@nil{\def\ef@argii{#2}%
    \ifx\ef@argii\ef@end\else\ifx\ef@linktxtcolor@set0%
    \def\ef@thislinkcolor{\@menucolor}\fi
    \expandafter\ef@gobbleToendmarker\fi}
%    \end{macrocode}
% Search for \texttt{/Launch}, if found, change the link color to |\@runcolor|
%    \begin{macrocode}
\def\ef@searchLaunch#1/Launch#2\@nil{\def\ef@argii{#2}%
    \ifx\ef@argii\ef@end\else\ifx\ef@linktxtcolor@set0%
    \def\ef@thislinkcolor{\@runcolor}\fi
    \expandafter\ef@gobbleToendmarker\fi}
%    \end{macrocode}
% Executed by |\eq@A|, which calls the search routines defined above, at least
% in the case of links. It also searches for |\rPage|.
%    \begin{macrocode}
\def\ef@preprocessA#1{%
    \let\rPage\relax\edef\ef@argi{#1}%
    \ifx\annot@type\annot@type@link
    \expandafter\ef@searchGoToR\ef@argi/GoToR\end\@nil
    \expandafter\ef@searchURI\ef@argi/URI\end\@nil
    \expandafter\ef@searchCmdURI\ef@argi\URI\end\@nil
    \expandafter\ef@searchNamed\ef@argi/Named\end\@nil
    \expandafter\ef@searchCmdNamed\ef@argi\Named\end\@nil
    \expandafter\ef@searchLaunch\ef@argi/Launch\end\@nil
    \ef@endmarker\fi
    \let\rPage\ef@rPage
    \expandafter\ef@searchrPage\ef@argi\rPage\end\@nil
}
%    \end{macrocode}
% \DescribeMacro{\A} This is the \emph{action dictionary} (used by links and forms). If the argument
% is empty, we do nothing, otherwise, we call |\ef@preprocessA|, then define
% the \texttt{/A <<...>>} dictionary.
%    \begin{macrocode}
\def\@eqA#1{\def\eq@arg{#1}\ifx\eq@arg\@empty
    \let\eq@A\@empty\else\ef@preprocessA{#1}%
    \def\eq@A{/A <<#1>>}\fi}\def\eq@A{}
%    \end{macrocode}
% \DescribeMacro{\mlLink} This is a key for the \cs{setLink} command. If
% we say \verb!\mlLink{true}! in the \cs{setLink} option list, we use
% \cs{mlhypertext} from \texttt{aeb\_mlink}, if that package is loaded.
%    \begin{macrocode}
\def\@eqmlLink#1{\def\eq@arg{#1}\ifx\eq@arg\ef@true
    \let\ef@mlLink=1\else\let\ef@mlLink=0\fi}
%    \end{macrocode}
% \DescribeMacro{\Lock} The Lock key is used with signature fields, currently, there is
% no nice user interface to this key. Typical entries are
%\begin{verbatim}
%   \Lock{/Action/All} % all fields in the doc
%   \Lock{/Action/Include % include all fields listed in Fields
%         /Fields [(field1)(field2)...]}
%   \Lock{/Action/Exclude % exclude all fields listed in Fields
%         /Fields [(field1)(field2)...]}
%\end{verbatim}
%    \begin{macrocode}
\def\@eqLock#1{\def\eq@arg{#1}\ifx\eq@arg\@empty
    \let\eq@Lock\@empty\else\def\eq@Lock{/Lock <<#1>>}\fi}
    \def\eq@Lock{}
%    \end{macrocode}
%\paragraph*{Additional Actions.}
% We search for \cs{AACalculate}.
% The \cs{ef@searchCalc} searches for \cs{AACalculate}. If, for some reason,
% \cs{AA\-Cal\-cu\-late} is within a pair of braces, this search will not find it,
% so don't group \cs{AACalculate} within braces, there is no reason to do so
% anyway. If we find \cs{AACalculate}, we set \cs{isCalculate} to \texttt{true}, which is normally
% \texttt{false}.
%    \begin{macrocode}
\newif\ifisCalculate\isCalculatefalse
\def\ef@searchCalc#1\AACalculate#2\@nil{%
    \ifx#2\end\else\aftergroup\isCalculatetrue\fi
}
%    \end{macrocode}
% \DescribeMacro{\AA} (02/06/09) The argument of \cs{@eqAA} is nonempty, we search for the token \cs{AACalculate}
% if found, we set \cs{ifisCalculate} to \texttt{true}. When the document author
% uses the \cs{ui}, the key \cs{AA} is not used, so this change does not affect
% this option. The user interface populates \cs{eq@AA} with a series of commands
% that are either empty or expand to the correct code.
%    \begin{macrocode}
\def\@eqAA#1{\def\eq@arg{#1}\ifx\eq@arg\@empty\let\eq@AA\@empty
    \else\begingroup\ef@searchCalc#1\AACalculate\end\@nil\endgroup
    \def\eq@AA{/AA <<#1>>}\fi}
%    \end{macrocode}
% Begin some additional action definitions for the user interface option
%    \begin{macrocode}
\def\eq@AA{/AA <<\eq@AAmouseup\eq@AAmousedown\eq@AAmouseenter%
    \eq@AAmouseexit\eq@AAonfocus\eq@AAonblur\eq@AAformat%
    \eq@AAkeystroke\eq@AAvalidate\eq@AAcalculate\eq@AApageopen%
    \eq@AApageclose\eq@AApagevisible\eq@AApageinvisible>>}
%    \end{macrocode}
%    \begin{macro}{AAmouseup}
%    \begin{macro}{AAmousedown}
%    \begin{macro}{AAmouseenter}
%    \begin{macro}{AAmouseexit}
%    \begin{macro}{AAonfocus}
%    \begin{macro}{AAonblur}
% These keys are generated internally and put into the parsing stream when
% the uses specifies actions using the user interface (|\ui|).
%    \begin{macrocode}
\def\@eqAAmouseup#1{\def\eq@arg{#1}\ifx\eq@arg\@empty
    \let\eq@AAmouseup\@empty\else\def\eq@AAmouseup{/U<<\JS{#1}>>}\fi}
\let\eq@AAmouseup\@empty
\def\@eqAAmousedown#1{\def\eq@arg{#1}\ifx\eq@arg\@empty
    \let\eq@AAmousedown\@empty\else
    \def\eq@AAmousedown{/D<<\JS{#1}>>}\fi}
\let\eq@AAmousedown\@empty
\def\@eqAAmouseenter#1{\def\eq@arg{#1}\ifx\eq@arg\@empty
    \let\eq@AAmouseenter\@empty\else
    \def\eq@AAmouseenter{/E<<\JS{#1}>>}\fi}
\let\eq@AAmouseenter\@empty
\def\@eqAAmouseexit#1{\def\eq@arg{#1}\ifx\eq@arg\@empty
    \let\eq@AAmouseexit\@empty\else
    \def\eq@AAmouseexit{/X<<\JS{#1}>>}\fi}
\let\eq@AAmouseexit\@empty
\def\@eqAAonfocus#1{\def\eq@arg{#1}\ifx\eq@arg\@empty
    \let\eq@AAonfocus\@empty\else
    \def\eq@AAonfocus{/Fo<<\JS{#1}>>}\fi}
\def\@eqAAmousefocus{\@eqAAonfocus}
\let\eq@AAonfocus\@empty
\def\@eqAAonblur#1{\def\eq@arg{#1}\ifx\eq@arg\@empty
    \let\eq@AAonblur\@empty\else
    \def\eq@AAonblur{/Bl<<\JS{#1}>>}\fi}
\def\@eqAAmouseblur{\def\@eqAAonblur}
\let\eq@AAonblur\@empty
\def\@eqAAformat#1{\def\eq@arg{#1}\ifx\eq@arg\@empty
    \let\eq@AAformat\@empty\else
    \def\eq@AAformat{/F<<\JS{#1}>>}\fi}
%    \end{macrocode}
%    \begin{macro}{AAformat}
%    \begin{macro}{AAkeystroke}
%    \begin{macro}{AAvalidate}
%    \begin{macro}{AAcalculate}
% Formatting, keystroke, validate, calculate tabs.
%    \begin{macrocode}
\let\eq@AAformat\@empty
\def\@eqAAkeystroke#1{\def\eq@arg{#1}\ifx\eq@arg\@empty
    \let\eq@AAkeystroke\@empty\else
    \def\eq@AAkeystroke{/K<<\JS{#1}>>}\fi}
\let\eq@AAkeystroke\@empty
\def\@eqAAvalidate#1{\def\eq@arg{#1}\ifx\eq@arg\@empty
    \let\eq@AAvalidate\@empty\else
    \def\eq@AAvalidate{/V<<\JS{#1}>>}\fi}
\let\eq@AAvalidate\@empty
\def\@eqAAcalculate#1{\def\eq@arg{#1}\ifx\eq@arg\@empty
    \let\eq@AAcalculate\@empty\else\isCalculatetrue
    \def\eq@AAcalculate{/C<<\JS{#1}>>}\fi}
\let\eq@AAcalculate\@empty
%    \end{macrocode}
%    \begin{macro}{AApageopen}
%    \begin{macro}{AApageclose}
%    \begin{macro}{AApagevisible}
%    \begin{macro}{AApageinvisible}
% Page related additional actions.
%    \begin{macrocode}
\def\@eqAApageopen#1{\def\eq@arg{#1}\ifx\eq@arg\@empty
    \let\eq@AApageopen\@empty\else
    \def\eq@AApageopen{/PO<<\JS{#1}>>}\fi}
\let\eq@AApageopen\@empty
\def\@eqAApageclose#1{\def\eq@arg{#1}\ifx\eq@arg\@empty
    \let\eq@AApageclose\@empty\else
    \def\eq@AApageclose{/PC<<\JS{#1}>>}\fi}
\let\eq@AApageclose\@empty
\def\@eqAApagevisible#1{\def\eq@arg{#1}\ifx\eq@arg\@empty
    \let\eq@AApagevisible\@empty\else
    \def\eq@AApagevisible{/PV<<\JS{#1}>>}\fi}
\let\eq@AApagevisible\@empty
\def\@eqAApageinvisible#1{\def\eq@arg{#1}\ifx\eq@arg\@empty
    \let\eq@AApageinvisible\@empty\else
    \def\eq@AApageinvisible{/PI<<\JS{#1}>>}\fi}
\let\eq@AApageinvisible\@empty
%    \end{macrocode}
%    \end{macro}
%    \end{macro}
%    \end{macro}
%    \end{macro}
%    \end{macro}
%    \end{macro}
%    \end{macro}
%    \end{macro}
%    \end{macro}
%    \end{macro}
%    \end{macro}
%    \end{macro}
%    \end{macro}
%    \end{macro}
%
% This ends the definitions use by the user interface option for additional actions.
%
% \paragraph*{Additional entries common to fields containing variable text:} \texttt{DR, DA, Q,
% DS, RV}.
%    \begin{macrocode}
%    \end{macrocode}
% \DescribeMacro{\DA} Default appearance (required)
%    \begin{macrocode}
\def\@eqDA#1{\def\eq@DA{#1}}
\def\eq@DA{\eq@textFont\space\eq@textSize\space Tf \eq@textColor}%
%    \end{macrocode}
% \DescribeMacro{\textFont} PDF form font
%    \begin{macrocode}
\def\@eqtextFont#1{\def\eq@textFont{/#1}}
\def\eq@textFont{/Helv}
%    \end{macrocode}
% \DescribeMacro{\textSize} PDF form text size
%    \begin{macrocode}
\def\@eqtextSize#1{\def\eq@textSize{#1}}
\def\eq@textSizeDefault{9}
\edef\eq@textSize{\eq@textSizeDefault}
%    \end{macrocode}
% \DescribeMacro{\RV} Rich text value
%\changes{v2.5l}{2011/01/28}{%
%   Wrap the \cs{RV} key in an XHTML \string\texttt{{\string\ltag}body\string\rtag} element, part of adding
%   better support for rich text strings for form fields.
%}
%    \begin{macrocode}
\def\eq@RV@Body{<?xml version="1.0"?><body %
    xfa:APIVersion="Acroform:2.7.0.0" %
    xfa:contentType="text/html" %
    xfa:spec="2.1" xmlns="http://www.w3.org/1999/xhtml" %
    xmlns:xfa="http://www.xfa.org/schema/xfa-data/1.0/">}
\def\eq@RV@endBody{</body>}
\def\@eqRV#1{\def\eq@arg{#1}\ifx\eq@arg\@empty
    \let\eq@RV\@empty\else\def\eq@RV{/RV(\eq@RV@Body#1%
    \eq@RV@endBody)\fi}}\def\eq@RV{}
%    \end{macrocode}
% \DescribeMacro{\DS} Rich text default style
%    \begin{macrocode}
\def\@eqDS#1{\def\eq@arg{#1}\ifx\eq@arg\@empty
    \let\eq@DS\@empty\else\def\eq@DS{/DS(#1)\fi}}\def\eq@DS{}
%    \end{macrocode}
% \DescribeMacro{\textColor} Text color
%    \begin{macrocode}
\def\@eqtextColor#1{\ef@parsePDFColor{#1}%
    \HyColor@IfXcolor{%
        \expandafter\HyColor@FieldColor%
        \expandafter{\ef@colorSpec@out}{\eq@textColor}{}{}%
    }{\edef\eq@textColor{\ef@colorSpec@out}}%
}
\def\eq@textColor{0 g}
%    \end{macrocode}
% \DescribeMacro{\Q} Quadding for text fields: \texttt{Q=0} left-justified, \texttt{Q=1} centered
%   \texttt{Q=2} right-justified.
%    \begin{macrocode}
\def\@eqQ#1{\def\eq@arg{#1}\ifx\eq@arg\@empty
    \let\eq@Q\@empty\else\def\eq@Q{/Q #1}\fi}\def\eq@Q{}
%    \end{macrocode}
% \paragraph*{Entries common to all fields:} \texttt{TU, Ff, V, DV, A, AA}
% \par\medskip\noindent
% \DescribeMacro{\DV}  The DV key sets the value of the field when the form field is reset.
% When the unicode option is taken (\cs{ifHy@unicode is \texttt{true}},
% we pass the argument through \cs{pdfstringdef} to get the octal encoding,
% which is the method hyperref uses.
% \changes{v2.7}{2014/07/08}{Removed the use of \string\cs{ifHy@unicode}, now pass all PDF text strings
% through \cs{pdfstringdef}.}
% \changes{v2.8a}{2015/07/12}{Added \string\cs{ef@isunicode} to automatically detect
% unicode}
% (2015/07/12) Added \cs{ef@isunicode} to automatically detect
% unicode. When the first token of the argument of \cs{@eqDV}
% and \cs{@eqV} is \cs{unicodeStr}, we bifurcate to \cs{@equDV}
% and \cs{@equV}, respectively.
%    \begin{macrocode}
\def\ef@isunicode#1\unicodeStr#2\@nil{\def\argi{#1}%
    \ifx\argi\@empty\def\ifbool@ef{iftrue}\else
    \def\ifbool@ef{iffalse}\fi}
\def\@eqDV#1{\ef@isunicode#1\unicodeStr\@nil
    \expandafter\csname\ifbool@ef\endcsname\@equDV{#1}\else
    \def\eq@arg{#1}\ifx\eq@arg\@empty
    \let\eq@DV\@empty\else
    \ef@pdfCRLFTABDefns\pdfstringdef\ef@uni@temp{#1}%
    \edef\eq@DV{/DV(\ef@uni@temp)}\makespecialJS\fi\fi}\def\eq@DV{}
%    \end{macrocode}
% \DescribeMacro{\nuDV} \cs{@eqnuDV} is the old definition of DV. This version does not
% use hyperref's unicode option. This version comes in handy
% in the acroflex package, where it is undesirable to unicode
% the default (and initial values).
%    \begin{macrocode}
\def\ef@pdfCRLFTABDefns{%
    \def\r{\textCR}\def\t{\textHT}\def\n{\textLF}}
\def\@eqnuDV#1{\def\eq@arg{#1}\ifx\eq@arg\@empty
    \let\eq@DV\@empty\else\def\eq@DV{/DV(#1)}\fi}
%    \end{macrocode}
% \DescribeMacro{\uDV} Unicoded DV, used in \textsf{acroflex.dtx}
%    \begin{macrocode}
\def\@equDV#1{\def\eq@arg{#1}\ifx\eq@arg\@empty
    \let\eq@DV\@empty\else\def\eq@DV{/DV<#1>}\fi}
%    \end{macrocode}
% \DescribeMacro{\V} |\V| is the field value (optional)
%    \begin{macrocode}
\def\@eqV#1{\ef@isunicode#1\unicodeStr\@nil
    \expandafter\csname\ifbool@ef\endcsname\@equV{#1}\else
    \def\eq@arg{#1}\ifx\eq@arg\@empty
    \let\eq@V\@empty\else
    \ef@pdfCRLFTABDefns\pdfstringdef\ef@uni@temp{#1}%
    \edef\eq@V{/V(\ef@uni@temp)}\makespecialJS\fi\fi}\def\eq@V{}
%    \end{macrocode}
% \DescribeMacro{\nuV} No Unicode V
%    \begin{macrocode}
\def\@eqnuV#1{\def\eq@arg{#1}\ifx\eq@arg\@empty
    \let\eq@V\@empty\else\def\eq@V{/V(#1)}\fi}
%    \end{macrocode}
% \DescribeMacro{\uV} Unicode version of V
%    \begin{macrocode}
\def\@equV#1{\def\eq@arg{#1}\ifx\eq@arg\@empty
    \let\eq@V\@empty\else\def\eq@V{/V<#1>}\fi}%
%    \end{macrocode}
% \DescribeMacro{\TU} Tool tip (optional, PDF 1.3)
%    \begin{macrocode}
\def\@eqTU#1{\ef@isunicode#1\unicodeStr\@nil
    \expandafter\csname\ifbool@ef\endcsname\@equTU{#1}\else
    \def\eq@arg{#1}\let\r@save\r\let\r\textCR
    \ifx\eq@arg\@empty\let\eq@TU\@empty\else
    \ef@pdfCRLFTABDefns\pdfstringdef\ef@uni@temp{#1}%
    \edef\eq@TU{/TU(\ef@uni@temp)}\makespecialJS\fi\fi
    \let\r\r@save}\def\eq@TU{}
%    \end{macrocode}
% \DescribeMacro{\uTU} Tool tip (optional, PDF 1.3), unicode version
%    \begin{macrocode}
\def\@equTU#1{\def\eq@arg{#1}\ifx\eq@arg\@empty
    \let\eq@TU\@empty\else\def\eq@TU{/TU<#1>}\fi}
%    \end{macrocode}
% \DescribeMacro{\Ff} The Field flags bit field, these values are listed below.
%    \begin{macrocode}
\def\@eqFf#1{\def\eq@arg{#1}\ifx\eq@arg\@empty
    \let\eq@Ff\@empty\else\def\eq@Ff{/Ff #1}\fi}
    \def\eq@Ff{}
%    \end{macrocode}
% \paragraph*{keys specific to text fields:} \texttt{MaxLen}
%    \begin{macrocode}
%    \end{macrocode}
% \DescribeMacro{\MaxLen} text fields only. Restricts number of characters
% input. Required if a comb field.
%    \begin{macrocode}
\def\@eqMaxLen#1{\def\eq@arg{#1}\ifx\eq@arg\@empty
    \let\eq@MaxLen\@empty\else\def\eq@MaxLen{/MaxLen #1}\fi}%
    \def\eq@MaxLen{}
%    \end{macrocode}
% \DescribeMacro{\H} Highlight, used in button fields and link annotation. Possible values
% are None, Push, Outline, Invert (respectively, \verb!\H{N}!, \verb!\H{P}!,
% \verb!\H{O}!, \verb!\H{I}!)
% The default highlighting is invert (I).
%    \begin{macrocode}
\def\@eqH#1{\def\eq@arg{#1}\ifx\eq@arg\@empty
    \let\eq@H\@empty\else\def\eq@H{/H/#1}\fi}\def\eq@H{}
%    \end{macrocode}
% \paragraph*{Appearance characteristics dictionary:}
% \texttt{MK=R, BC, BG, CA, RC, AC, I, RI, IX, IF, TP}
%    \begin{macrocode}
%    \end{macrocode}
% \DescribeMacro{\MK} A dictionary containing other keys
%    \begin{macrocode}
\def\@eqMK#1{\def\eq@arg{#1}\ifx\eq@arg\@empty
    \let\eq@MK\@empty\else\def\eq@MK{/MK << #1 >> }\fi}%
    \def\eq@MK{}
%    \end{macrocode}
% \DescribeMacro{\R} Rotation of field, values 0, 90, 180, 270.
%    \begin{macrocode}
    \let\@vertRotate=0
    \def\@eqR#1{\def\eq@R@value{#1}\ifx\eq@R@value\@empty
        \let\eq@R\@empty\else
%    \end{macrocode}
% Determine if we are rotating 90 or 270, if so, let a weak switch
% \cs{@vertRotate} to 1
%    \begin{macrocode}
        \@tempcnta=\eq@R@value\relax
        \ifnum\@tempcnta<0 \@tempcnta=-\@tempcnta\fi
        \ifnum\@tempcnta=0 \else\ifnum\@tempcnta=180 \else
            \let\@vertRotate=1\fi\fi
        \def\eq@R{/R #1}\fi}
    \def\eq@R{}
%    \end{macrocode}
%   \DescribeMacro{\BC} Boundary color
% \changes{v2.5j}{2011/01/18 }{%
%    Changed \string\cs{def}\string\cs{eq@arg} to \string\cs{edef}\string\cs{eq@arg} in the definition of
%    \string\cs{@eqBC} and \string\cs{@eqBG}. When the argument is a macro that is empty,
%    we can exceed {\string\TeX} capacity.
%}
%    \begin{macrocode}
    \def\@eqBC#1{\edef\eq@arg{#1}\ifx\eq@arg\@empty
        \let\eq@BC\@empty\else % 2010/07/23
        \expandafter\ef@isitnamed\eq@arg\ef@nil
        \ifx\ef@latex@color\ef@y\expandafter
            \HyColor@XZeroOneThreeFour\expandafter{\eq@arg}{\eq@BC}{}{}%
            \edef\eq@BC{/BC [\eq@BC]}\else
            \def\eq@BC{/BC [#1]}\fi
    \fi}
    \def\eq@BC{}
%    \end{macrocode}
% \DescribeMacro{\BG} Background color
%    \begin{macrocode}
    \def\@eqBG#1{\edef\eq@arg{#1}\ifx\eq@arg\@empty
        \let\eq@BG\@empty\else % 2010/07/23
        \expandafter\ef@isitnamed\eq@arg\ef@nil
        \ifx\ef@latex@color\ef@y\expandafter
            \HyColor@XZeroOneThreeFour\expandafter{\eq@arg}{\eq@BG}{}{}%
            \edef\eq@BG{/BG [\eq@BG]}\else
            \def\eq@BG{/BG [#1]}\fi
        \fi}
    \def\eq@BG{}
%    \end{macrocode}
% \DescribeMacro{\CA} normal appearance text
%    \begin{macrocode}
    \def\@eqCA#1{\ef@isunicode#1\unicodeStr\@nil
    \expandafter\csname\ifbool@ef\endcsname\@equCA{#1}\else
        \def\eq@arg{#1}\ifx\eq@arg\@empty
        \let\eq@CA\@empty\let\ef@kvCA\@empty
        \else\ef@pdfCRLFTABDefns
        \pdfstringdef\ef@uni@temp{#1}%
        \def\eq@CA{#1}\edef\ef@kvCA{/CA(\ef@uni@temp)}%
        \makespecialJS\fi\fi}
    \def\eq@CA{}\def\ef@kvCA{}
%    \end{macrocode}
% \DescribeMacro{\uCA} normal appearance text, unicode version
%    \begin{macrocode}
    \def\@equCA#1{\def\eq@arg{#1}\ifx\eq@arg\@empty
        \let\eq@CA\@empty\let\ef@kvCA\@empty
        \else\def\eq@CA{#1}\def\ef@kvCA{/CA<#1>}\fi}
%    \end{macrocode}
% \DescribeMacro{\RC} Roll over text
%    \begin{macrocode}
    \def\@eqRC#1{\ef@isunicode#1\unicodeStr\@nil
        \expandafter\csname\ifbool@ef\endcsname\@equRC{#1}\else
        \def\eq@arg{#1}\ifx\eq@arg\@empty
        \let\eq@RC\@empty\let\ef@kvRC\@empty
        \else\ef@pdfCRLFTABDefns
        \pdfstringdef\ef@uni@temp{#1}%
        \def\eq@RC{#1}\edef\ef@kvRC{/RC(\ef@uni@temp)}%
        \makespecialJS\fi\fi}
    \def\eq@RC{}\def\ef@kvRC{}
%    \end{macrocode}
% \DescribeMacro{\uRC} Roll over text, unicode version
%    \begin{macrocode}
    \def\@equRC#1{\def\eq@arg{#1}\ifx\eq@arg\@empty
        \let\eq@RC\@empty\let\ef@kvRC\@empty
        \else\def\eq@RC{#1}\def\ef@kvRC{/RC<#1>}\fi}
%    \end{macrocode}
% \DescribeMacro{\AC} Push text
%    \begin{macrocode}
    \def\@eqAC#1{\ef@isunicode#1\unicodeStr\@nil
        \expandafter\csname\ifbool@ef\endcsname\@equAC{#1}\else
        \def\eq@arg{#1}\ifx\eq@arg\@empty
        \let\eq@AC\@empty\let\ef@kvAC\@empty
        \else\ef@pdfCRLFTABDefns
        \pdfstringdef\ef@uni@temp{#1}%
        \def\eq@AC{#1}\edef\ef@kvAC{/AC(\ef@uni@temp)}%
        \makespecialJS\fi\fi}
    \def\eq@AC{}\def\ef@kvAC{}
%    \end{macrocode}
% \DescribeMacro{\uAC} Push text, unicode version
%    \begin{macrocode}
    \def\@equAC#1{\def\eq@arg{#1}\ifx\eq@arg\@empty
        \let\eq@AC\@empty\let\ef@kvAC\@empty
        \else\def\eq@AC{#1}\def\ef@kvAC{/AC<#1>}\fi}
%    \end{macrocode}
% Other keys of \texttt{MK} include: \texttt{I}, \texttt{RI}, \texttt{IX}, \texttt{IF} and \texttt{TP}
% If I haven't covered everything, use this macro to insert
% material into \texttt{MK}\par\medskip\noindent
% (02/07/09) We begin support for the \texttt{MK} dictionary for a push
% button with an icon appearance, the entries in the \texttt{MK} dictionary that
% effect an icon of the push button has this form:
%\begin{verbatim}
%   \ifx\eq@AP\@empty
%       /MK <<\eq@R\eq@BC\eq@BG%
%           \ef@kvCA\ef@kvRC\ef@kvAC\eq@IconMK\eq@mkIns>>
%   \else
%       \eq@AP
%   \fi
%\end{verbatim}
% The \cs{eq@IconMK} macro inserts the appropriate code for an icon appearance.
%    \begin{macrocode}
%    \end{macrocode}
% \cs{eq@define@IconMK} defines the elements of the \texttt{MK} dictionary,
% used only if there is an icon appearance. The default definition of
% \cs{eq@IconMK} is empty.
%    \begin{macrocode}
\def\eq@define@IconMK{\def\eq@IconMK{\eq@I\eq@RI\eq@IX\eq@TP
    /IF<<\eq@SW\eq@ST\eq@PA\eq@FB>>}}
\let\eq@IconMK\@empty
%    \end{macrocode}
% \DescribeMacro{\I} an indirect reference to a form XObject defining the
% buttons's \emph{normal icon}
% \changes{v2.9}{2016/05/09}{Modified \string\cs{I}, \string\cs{RI}, and
% \string\cs{IX} to accommodate pdftex for null argument.}
%    \begin{macrocode}
\ifpdf\def\eq@relRef#1{0 0 R}\else %{#1\space 0 R}
    \ifxetex\def\eq@relRef#1{#1}\else
    \def\eq@relRef#1{{#1}}\fi\fi
\def\ef@null{null}
\def\@eqI#1{\ifx\annot@type\annot@type@button
    \def\eq@arg{#1}\ifx\eq@arg\@empty
    \let\eq@I\@empty\else
    \def\eq@I{/I \eq@relRef{#1}}%
    \ifpdf\ifx\eq@arg\ef@null
    \def\eq@I{/I 0 0 R}\fi\fi
    \eq@define@IconMK\fi\fi}
\def\eq@I{}
%    \end{macrocode}
% \DescribeMacro{\RI} an indirect reference to a form XObject defining
% the buttons's \emph{rollover icon}
%    \begin{macrocode}
\def\@eqRI#1{\ifx\annot@type\annot@type@button
    \def\eq@arg{#1}\ifx\eq@arg\@empty
    \let\eq@RI\@empty\else
    \def\eq@RI{/RI \eq@relRef{#1}}%
    \ifpdf\ifx\eq@arg\ef@null
    \def\eq@RI{/RI 0 0 R}\fi\fi
    \eq@define@IconMK\fi\fi}
\def\eq@RI{}
%    \end{macrocode}
% \DescribeMacro{\IX} an indirect reference to a form XObject defining
% the buttons's \emph{down icon}
%    \begin{macrocode}
\def\@eqIX#1{\ifx\annot@type\annot@type@button
    \def\eq@arg{#1}\ifx\eq@arg\@empty
    \let\eq@IX\@empty\else
    \def\eq@IX{/IX \eq@relRef{#1}}%
    \ifpdf\ifx\eq@arg\ef@null
    \def\eq@IX{/IX 0 0 R}\fi\fi
    \eq@define@IconMK\fi\fi}
\def\eq@IX{}
%    \end{macrocode}
% \DescribeMacro{\TP} A code indicating the \texttt{layout} of the text and icon; these codes are
%        0 (label only); 1 (icon only); 2 (label below icon); 3 (label above icon); 4 (label to the right of icon);
%        5 (label to the left of icon); 6 (label overlaid on the icon). The default is 0.
%    \begin{macrocode}
\def\@eqTP#1{\def\eq@arg{#1}\ifx\eq@arg\@empty
    \let\eq@TP\@empty\else\def\eq@TP{/TP #1}\fi}
\def\eq@TP{/TP 0} % default 0
%    \end{macrocode}
% \DescribeMacro{\SW} The \emph{scale when key}. Permissible values are \texttt{A} (always scale),
%   \texttt{B} (scale when icon is too big), \texttt{S} (scale when icon is too small), \texttt{N}
%   (never scale). The default is \texttt{A}.
%    \begin{macrocode}
\def\@eqSW#1{\def\eq@arg{#1}\ifx\eq@arg\@empty
    \let\eq@SW\@empty\else\def\eq@SW{/SW/#1}\fi}
\def\eq@SW{/SW/A} % the default, always scale
%    \end{macrocode}
% \DescribeMacro{\ST} The \emph{scaling type.} Permissible values are \texttt{A}
%    (anamorphic scaling); \texttt{P} (proportional scaling). The default is \texttt{P}.
%    \begin{macrocode}
\def\@eqST#1{\def\eq@arg{#1}\ifx\eq@arg\@empty
    \let\eq@ST\@empty\else\def\eq@ST{/S/#1}\fi}
\def\eq@ST{/S/P} % the default, proportional scaling
%    \end{macrocode}
% \DescribeMacro{\PA} The \textit{position array.} An array of two numbers, each
%  between 0 and 1 indicating the fraction of left-over space to allocate at the left and bottom
%  of the annotation rectangle. The two numbers should be separated by a space. The default value, \verb!\PA{.5 .5}!
%    \begin{macrocode}
\def\@eqPA#1{\def\eq@arg{#1}\ifx\eq@arg\@empty
    \let\eq@PA\@empty\else\def\eq@PA{/A [#1]}\fi}
\def\eq@PA{/A [0.5 0.5]} % the default
%    \end{macrocode}
% \DescribeMacro{\FB} The \emph{fit bounds} Boolean. If \texttt{true}, the button appearance
%   is scaled to fit fully within the bounds of the annotation without taking into consideration
%   the line width of the border. The default is \texttt{false}.
%    \begin{macrocode}
\def\@eqFB#1{\def\eq@arg{#1}\ifx\eq@arg\@empty
    \let\eq@PA\@empty\else\def\eq@FB{/FB #1}\fi}
\def\eq@FB{/FB false} % the default
%    \end{macrocode}
% \DescribeMacro{\mkIns} used for miscellaneous entries for \texttt{MK} dictionary.
%    \begin{macrocode}
\def\@eqmkIns#1{\def\eq@mkIns{#1}}\def\eq@mkIns{}
%    \end{macrocode}
% \paragraph*{Additional entries specific to choice fields:} \texttt{Opt, TI, I}
%    \begin{macrocode}
% an array of options in the list
\def\@eqOpt#1{\def\eq@arg{#1}\ifx\eq@arg\@empty
    \let\eq@Opt\@empty\else\def\eq@Opt{/Opt [#1]}\fi}
    \def\eq@Opt{}
% for scrollable list boxes, the top index.
\def\@eqTI#1{\def\eq@arg{#1}\ifx\eq@arg\@empty
    \let\eq@TI\@empty\else\def\eq@TI{/TI #1}\fi}
    \def\eq@TI{}
%    \end{macrocode}
% When all else fails, use the \cs{rawPDF} command to modify the widget.
%    \begin{macrocode}
\def\@eqrawPDF#1{\def\eq@rawPDF{#1}}\def\eq@rawPDF{}
%    \end{macrocode}
% The following is in support for multi-line links
%    \begin{macrocode}
%    \end{macrocode}
% \DescribeMacro{\QuadPoints} Used by \texttt{aeb\_mlink}, used internally by that
% package to create multi-line links.
%    \begin{macrocode}
\def\@eqQuadPoints#1{\def\eq@arg{#1}\ifx\eq@arg\@empty
    \let\eq@QuadPoints\@empty\else
    \def\eq@QuadPoints{/QuadPoints {#1}}\fi}
\def\eq@QuadPoints{}
%    \end{macrocode}
% \DescribeMacro{\Color} Changed |\def\eq@arg| to |\edef\eq@arg| (01/18/11)
%    \begin{macrocode}
\def\@eqColor#1{\edef\eq@arg{#1}\ifx\eq@arg\@empty
    \let\eq@Color\@empty\else
    \HyColor@XZeroOneThreeFour{#1}{\eq@Color}{}{}%
    \edef\eq@Color{/C[\eq@Color]}\fi}
\def\eq@Color{}
%    \end{macrocode}
% \DescribeMacro{linktxtcolor} key to set the color of the link through
% the option list.
%    \begin{macrocode}
\def\@eqlinktxtcolor#1{%
    \def\ef@argi{#1}\ifHy@colorlinks
        \ifx\ef@argi\@empty\let\ef@colorthislink\normalcolor\else
        \let\ef@linktxtcolor@set=1\def\ef@thislinkcolor{#1}\fi\fi
}\let\ef@linktxtcolor@set=0
%    \end{macrocode}
% \paragraph*{Specialized, non-PDF Spec, commands}
%    \begin{macrocode}
%    \end{macrocode}
%    \begin{macro}{\rectH}
%    \begin{macro}{\rectW}
% Use to set the height and width of a widget.
%    \begin{macrocode}
\def\@eqrectH#1{\def\eq@rectH{#1}\ifx\eq@rectH\@empty\else
    \setlength\eflength\eq@rectH\edef\eq@rectH{\the\eflength}\fi}
\def\@eqrectW#1{\def\eq@rectW{#1}\ifx\eq@rectW\@empty\else
    \setlength\eflength\eq@rectW\edef\eq@rectW{\the\eflength}\fi}
%    \end{macrocode}
%    \end{macro}
%    \end{macro}
% \DescribeMacro{\objdef} Insert an indirect reference (ps only drivers). The value of this
% key must be unique throughout the whole document.
% This is a \textbf{pdfmark} feature that inserts a references to this COS object,
% used with setting the tab order using the structure.
%    \begin{macrocode}
\def\@eqobjdef#1{\def\ef@arg{#1}\ifx\ef@arg\@empty
    \let\eq@objdef\@empty\else\def\eq@objdefName{#1}%
    \def\eq@objdef{/_objdef {#1}}\fi
}
\let\eq@objdef\@empty
%    \end{macrocode}
% \DescribeMacro{\taborder} The \cs{taborder} key is used for the
% \texttt{dvipsone}/\texttt{dvips} options on a page where the tab order is
% determined through structure.
%    \begin{macrocode}
\def\@eqtaborder#1{\def\ef@arg{#1}\ifx\ef@arg\@empty
    \let\eq@taborder\@empty\else
    \def\eq@taborder{#1}\fi
}
\let\eq@taborder\@empty
%    \end{macrocode}
% \DescribeMacro{\autoCenter} Auto-center feature, values are
% |\autoCenter{y}| (the default) or |\antoCenter{n}|.
%    \begin{macrocode}
\def\ef@y{y}\def\ef@n{n}
\def\@eqautoCenter#1{\def\ef@arg{#1}\ifx\ef@arg\ef@y
    \let\autoCenter\ef@y\else\ifx\ef@arg\ef@n
    \let\autoCenter\ef@n\else\let\autoCenter\ef@y
    \PackageWarning{eforms}{The value of '#1' is not a
    supported value\MessageBreak for \string\autoCenter.\MessageBreak
    Using the default of 'y'}%
    \fi\fi}
\let\autoCenter\ef@y
%    \end{macrocode}
% \DescribeMacro{\inline} The if |\inline{y}|, then we attempt to
% get a better vertical positioning. Designed for inline form fields.
% \changes{v2.5h}{2012/11/17}{Introduce the \string\cs{inline} key designed for
% inline form fields.}
%    \begin{macrocode}
\newif\ifeq@inlineCenter \eq@inlineCenterfalse
\let\inlineCenter=n
\def\@eqinline#1{\eq@inlineCenterfalse
    \def\ef@arg{#1}\ifx\ef@arg\ef@y
    \let\inlineCenter\ef@y\eq@inlineCentertrue\else
    \ifx\ef@arg\ef@n\let\inlineCenter\ef@n\else\let\inlineCenter\ef@n
    \PackageWarning{eforms}{The value of '#1' is not a
    supported value\MessageBreak for \string\inline.\MessageBreak
    Using the default of 'n'}%
    \fi\fi}
%    \end{macrocode}
% \DescribeMacro{\symbolchoice} The symbol used for a check box or radio button.
% Elsewhere, we have defined,
%\begin{verbatim}
%   \def\eq@check{4}
%   \def\eq@circle{l}
%   \def\eq@cross{8}
%   \def\eq@diamond{u}
%   \def\eq@square{n}
%   \def\eq@star{H}
%\end{verbatim}
% Possible values for this key are \texttt{check}, \texttt{circle},
% \texttt{cross}, \texttt{diamond}, \texttt{square}, and \texttt{star}.
%    \begin{macrocode}
\def\@eqsymbolchoice#1{\expandafter\ifx\csname eq@#1\endcsname\relax
   \typeout{exerquiz: `#1' is not an acceptable option
   for \string\symbolechoice, inserting default, `check'.}
   \edef\symbol@choice{\eq@check}\else
%    \end{macrocode}
% We take \texttt{\#1} and form the command \cs{eq@\#1}, to match one of
% the definitions listed above.
%    \begin{macrocode}
   \edef\symbol@choice{\csname eq@#1\endcsname}\fi
}
%    \end{macrocode}
% \DescribeMacro{\protect} A protect feature for protecting a key from begin changed
% by the user through the optional arguments.
%    \begin{macrocode}
\def\eq@protect#1{\let#1\@empty}
%    \end{macrocode}
%
% \subsubsection{Support for Hex escapes in PDF names}
%    \begin{macrocode}
\begingroup\catcode`\#=12 \catcode`*=6
\gdef\HEXNAME{efHex}
\ifpdfmarkup\gdef\ef@Hx*1*2{\@nameuse{efHex*1*2}}\else
\gdef\ef@Hx*1*2{#*1*2}\fi
\endgroup
\ifpdfmarkup
\def\HexGlyph#1#2{\global\@namedef{efHex#1}{#2}}
\def\ef@inputPDFHEX{\InputIfFileExists{pdfdochex.def}
    {\typeout{Inputtingpdfdochex.def}}{}}
\expandafter\ef@inputPDFHEX\fi
%    \end{macrocode}
% \subsubsection{Parsing PDF Color}
%
% The command is called by the \cs{textColor} key of a form field,
% the return value, \cs{ef@colorSpec@out}, is then used in the color
% specification of the text. If \textsf{xcolor} is loaded, we pass
% the named color or latex color specification with color model back,
% and is turned processed by the \textsf{hycolor} command \cs{HyColor@FieldColor}.
%    \begin{macrocode}
\def\ef@semicolon{;}
\def\ef@stripsemi#1;\@nil{\def\ef@colorSpec@out{#1}}
%    \end{macrocode}
%    \begin{macrocode}
\def\ef@isitnamed{\let\ef@latex@color\ef@y
    \@ifnextchar[{\ef@gobbletonil}{%
    \@tfor\mytok:=.0123456789\do{%
        \if\mytok\@let@token
        \let\ef@latex@color\ef@n
        \@break@tfor\fi}\ef@gobbletonil}}
%    \end{macrocode}
%    \begin{macrocode}
%\def\ef@getfirst{\@ifnextchar[{\ef@gobbletonil}{%
%    \@tfor\mytok:=.0123456789\do{%
%        \if\mytok\@let@token
%        \edef\ef@colorSpec@out{[gray]\ef@colorSpec@out}%
%        \@break@tfor\fi}\ef@gobbletonil}}
\def\ef@gobbletonil#1\ef@nil{}
%    \end{macrocode}
%    \begin{macro}{\ef@parsePDFColor}
% \cs{ef@parsePDFColor} tries to guarantee backward compatibility, and tries to
% assure the color passed to \textsf{hycolor} fits its design expectations.
% We begin by completely expanding the argument, for the argument may be in
% macro form.
%    \begin{macrocode}
\def\ef@parsePDFColor#1{\edef\ef@color@arg{#1}\ef@parsePDFColori}
\def\ef@parsePDFColori{%
    \expandafter\ef@@parsePDFColor\ef@color@arg; ; ; ; ; ;\\}
\def\ef@@parsePDFColor#1 #2 #3 #4 #5 #6\\{%
%    \end{macrocode}
% \paragraph*{Test of gray or model or named.} We test whether this is either a named color, or gray color space. This is
% the case if \texttt{\#3} is either `\texttt{g;}' or `\texttt{;}'.
%    \begin{macrocode}
    \def\argii{#2}\def\ef@cmp{g;}%
    \ifx\argii\ef@cmp
%    \end{macrocode}
% It is a PDF color \texttt{<num> g}, if \textsf{xcolor} is loaded we pass it as
% \verb![gray]{#1}!; otherwise, we return \texttt{\#1 g}.
%    \begin{macrocode}
        \expandafter\ifx\csname convertcolorspec\endcsname\relax
            \def\ef@colorSpec@out{#1 g}\else
            \def\ef@colorSpec@out{[gray]{#1}}%
        \fi
    \else\ifx\argii\ef@semicolon
%    \end{macrocode}
% Either a named color or a single number
%    \begin{macrocode}
        \expandafter\ifx\csname convertcolorspec\endcsname\relax
            \ef@isitnamed#1\ef@nil
            \ifx\ef@latex@color\ef@n
                \ef@stripsemi#1\@nil
                \edef\ef@colorSpec@out{\ef@colorSpec@out\space g}%
            \else
                \ef@stripsemi#1\@nil
                \PackageWarning{eforms}{Color specification
                `\ef@colorSpec@out' not supported\MessageBreak
                without xcolor, using a black color}
                \def\ef@colorSpec@out{0 g}%
            \fi
        \else  % xcolor
%    \end{macrocode}
% If \textsf{xcolor} is not loaded, we do nothing; if \texttt{xcolor} is loaded
% we determine if the first token is a `\texttt{[}' indicating a color space
% specification; or if the first token is a number, indicating that this is a number.
% If neither cases are detected, we assume a named color.
%    \begin{macrocode}
            \ef@isitnamed#1\ef@nil
            \ifx\ef@latex@color\ef@n
                \ef@stripsemi#1\@nil
                \edef\ef@colorSpec@out{[gray]{\ef@colorSpec@out}}%
            \else
                \ef@stripsemi#1\@nil
                \edef\ef@colorSpec@out{\ef@colorSpec@out}%
            \fi
        \fi
    \else % not semicolon
        \def\argiv{#4}\def\ef@cmp{rg;}%
%    \end{macrocode}
% \paragraph*{RGB test.} If \texttt{\#4} is either `\texttt{rg;}' or `\texttt{;}', we are RGB
%    \begin{macrocode}
        \ifx\argiv\ef@cmp
            \expandafter\ifx\csname convertcolorspec\endcsname\relax
                \def\ef@colorSpec@out{#1 #2 #3 rg}\else
                \def\ef@colorSpec@out{[rgb]{#1,#2,#3}}\fi
        \else\ifx\argiv\ef@semicolon
            \expandafter\ifx\csname convertcolorspec\endcsname\relax
            \ef@stripsemi#1 #2 #3\@nil
            \edef\ef@colorSpec@out{\ef@colorSpec@out\space rg}\else
            \ef@stripsemi#3\@nil
            \edef\ef@colorSpec@out{[rgb]{#1,#2,\ef@colorSpec@out}}\fi
%                Help!: {#1,#2,\ef@colorSpec@out}%
        \else
            \def\argv{#5}\edef\ef@cmp{k;}
%    \end{macrocode}
% \paragraph*{CMYK test.} If \texttt{\#5} is either `\texttt{k;}' or `\texttt{;}', we are CMYK.
%    \begin{macrocode}
        \ifx\argv\ef@cmp
            \expandafter\ifx\csname convertcolorspec\endcsname\relax
                \def\ef@colorSpec@out{#1 #2 #3 #4 k}\else
                \def\ef@colorSpec@out{[cmyk]{#1,#2,#3,#4}}\fi
        \else
            \ifx\argv\ef@semicolon
                \ef@stripsemi#1 #2 #3 #4\@nil
                \expandafter\ifx\csname convertcolorspec\endcsname\relax
                \edef\ef@colorSpec@out{\ef@colorSpec@out\space k}\else
                \ef@stripsemi#4\@nil
                \edef\ef@colorSpec@out{%
                    [cmyk]{#1,#2,#3,\ef@colorSpec@out}}\fi
        \else\ef@parseColor@iv
    \fi\fi\fi\fi\fi\fi
}
%    \end{macrocode}
%    \end{macro}
% Error messages for color spec parsing
%    \begin{macrocode}
\def\ef@parseColor@iv{\PackageError{AeB}{%
    The number of arguments
    is incorrect.\MessageBreak I was expecting
    1, 3, or 4 components of color}{Specify the correct number of
    components for the color space.}}
%    \end{macrocode}
%
% \subsection{Support for setting the calculation order}
% The command \cs{calcOrder}\DescribeMacro{\calcOrder} set the order of calculation
% of all calculation fields listed in this comma delimited list.
%\begin{verbatim}
%   \calcOrder{field3,field2,field1}
%\end{verbatim}
%    \begin{macrocode}
\def\calcOrder#1{\let\efCalcOrder\@gobble
    \@for\coi:=#1\do{\edef\efCalcOrder{\efCalcOrder,"\coi"}}%
    \edef\efCalcOrder{[\efCalcOrder]}}
\@onlypreamble\calcOrder
\def\efCalcOrder{[]}
%    \end{macrocode}
%
%\subsection{Symbol Definitions}
%
% Some definitions for radio fields and checkboxes
%    \begin{macrocode}
\def\eq@check{4}
\def\eq@circle{l}
\def\eq@cross{8}
\def\eq@diamond{u}
\def\eq@square{n}
\def\eq@star{H}
%    \end{macrocode}
%    \begin{macro}{\symbolchoice}
% Use this macro to change the symbol used in radio and
% checkboxes.  The default is \cmd{\eq@check}.  This macro takes one
% argument: permissible values are: check, circle, cross, diamond,
% square, and star. The definition of \cs{symbolchoice} is given
% above in the definition of \cs{@eqsymbolchoice}.
%    \begin{macrocode}
\let\symbolchoice\@eqsymbolchoice
%    \end{macrocode}
% Set the default to `check'.
%    \begin{macrocode}
\symbolchoice{check}
%    \end{macrocode}
%    \end{macro}
%\subsection{Convenience Commands}
%
%\subsubsection{Writing Actions}
%
% Writing actions for eForms requires certain key-value combinations. The following
% commands provides the correct syntax, the code is inserted via the required
% argument of each.
%
%    \begin{macro}{\JS}
%    \begin{macro}{\Named}
%    \begin{macro}{\URI}
%    \begin{macro}{\Next}
%    \begin{macro}{\toggleAttachmentsPanel}
% Convenience commands for writing JavaScript and for executing named events.
%    \begin{macrocode}
\providecommand{\JS}[1]{/S/JavaScript/JS(#1)}
\newcommand{\URI}[1]{/S/URI/URI(#1)}
\providecommand{\Named}[1]{/S/Named/N/#1}
\newcommand{\Next}[1]{/Next<<#1>>}
\providecommand{\toggleAttachmentsPanel}[2]{%
    \setLink[\Border{0 0 0}\A{\Named{ShowHideFileAttachment}}]
    {\textcolor{#1}{#2}}}%
%    \end{macrocode}
%    \end{macro}
%    \end{macro}
%    \end{macro}
%    \end{macro}
%    \end{macro}
% When entering triggers into the AA dictionary, use these commands (all the ones of
% the form \cs{AA<Name>}. These all have one argument, the action to take, usually
% a JavaScript action.  There is now a routine in the parsing of the AA dictionary
% that searches for \cs{AACalculate}, if found, sets the switch \cs{ifisCalculate}.
% For this search to succeed, these helper macros must be used.
%    \begin{macro}{\AAMouseUp}
%    \begin{macro}{\AAMouseDown}
%    \begin{macro}{\AAMouseEnter}
%    \begin{macro}{\AAMouseExit}
%    \begin{macro}{\AAOnFocus}
%    \begin{macro}{\AAOnBlur}
% This set of six appear in the Action tab of the field properties dialog box.
% Here, we don't assume the actions are JavaScript actions. The \cs{AAMouseUp}
% key is normally not used, rather, the mouse up action can be define using
% the \cs{A} key, such as \verb!\A{\JS{app.beep(0)}}!
%    \begin{macrocode}
\newcommand{\AAMouseUp}[1]{/U<<#1>>}
\newcommand{\AAMouseDown}[1]{/D<<#1>>}
\newcommand{\AAMouseEnter}[1]{/E<<#1>>}
\newcommand{\AAMouseExit}[1]{/X<<#1>>}
\newcommand{\AAOnFocus}[1]{/Fo<<#1>>}
\newcommand{\AAOnBlur}[1]{/Bl<<#1>>}
%    \end{macrocode}
%    \end{macro}
%    \end{macro}
%    \end{macro}
%    \end{macro}
%    \end{macro}
%    \end{macro}
%    \begin{macro}{\AAFormat}
%    \begin{macro}{\AAKeystroke}
%    \begin{macro}{\AAValidate}
%    \begin{macro}{\AACalculate}
% These four triggers  are JavaScript only, so we use \cs{JS}
% to insert the appropriate code. In the user interface, they appear
% in the Format, Validate, and Calculate tabs of the text field, and the
% combo box.
%    \begin{macrocode}
\newcommand{\AAFormat}[1]{/F<<\JS{#1}>>}
\newcommand{\AAKeystroke}[1]{/K<<\JS{#1}>>}
\newcommand{\AAValidate}[1]{/V<<\JS{#1}>>}
\newcommand{\AACalculate}[1]{/C<<\JS{#1}>>}
%    \end{macrocode}
%    \end{macro}
%    \end{macro}
%    \end{macro}
%    \end{macro}
%    \begin{macro}{\AAPageOpen}
%    \begin{macro}{\AAPageClose}
%    \begin{macro}{\AAPageVisible}
%    \begin{macro}{\AAPageInvisible}
% Page actions associated with a form field (PDF 1.5 or later)
% Originally designed for multimedia annotations, but can be
% use for form fields as well. There is no UI for these JavaScripts. The
% order of execution of these is Page 1: PV, Page 1: PO, Page 2: PV,
% Page 1: PC, Page 2: PO.
%\changes{v1.0b}{2006/10/14}
%{
%    Added page actions for annotations, PDF 1.5 or later
%}
%    \begin{macrocode}
\newcommand{\AAPageOpen}[1]{/PO<<\JS{#1}>>}
\newcommand{\AAPageClose}[1]{/PC<<\JS{#1}>>}
\newcommand{\AAPageVisible}[1]{/PV<<\JS{#1}>>}
\newcommand{\AAPageInvisible}[1]{/PI<<\JS{#1}>>}
%    \end{macrocode}
%    \end{macro}
%    \end{macro}
%    \end{macro}
%    \end{macro}
%\subsubsection{Saving Paths}
%    \begin{macro}{\definePath}
% A convenience command for saving a path or an URL, usage,
%\begin{verbatim}
%\definePath{\myPath}{http://www.example.edu/~dpstory}
% ...
%\setLink[\A{\URI{\myPath}}]{Go There!}
%\end{verbatim}
%    \begin{macrocode}
\newcommand{\definePath}[1]{\def\ef@ctrlName{#1}%
    \hyper@normalise\ef@definePath}
\def\ef@definePath#1{\expandafter\xdef\ef@ctrlName{#1}}
%    \end{macrocode}
%    \end{macro}
%
% \subsection{Annotation and Field Flags}
%
% \subsubsection{Annotation Flag \texttt{/F} Definitions}\label{F}
%
%    \begin{macrocode}
\def\FHidden{2}         % bit 2: hidden field
\def\FPrint{4}          % bit 3: print (we set this bit by default)
\def\FNoPrint{-4}       % bit 3: -print (this clears the bit)
\def\FNoView{32}        % bit 6: no view
\def\FLock{128}         % bit 8: locked field (PDF 1.4)
%    \end{macrocode}
%\paragraph*{Notes:}\par
%\noindent\begin{tabular}{ll}
% Visible (and printable)       &|\F\FPrint|\\
%Hidden but printable          &|\F\FNoView\F\FPrint|\\
%Visible but doesn't print     &|\F\FNoPrint|\\
%Hidden (and does not print)   &|\F\FHidden\F\FPrint|
%\end{tabular}
%
%\subsubsection{Field Flags /Ff Definitions}\label{Ff}
%    \begin{macrocode}
\def\FfReadOnly{1}                   % all
\def\FfRequired{2}                   % all
\def\FfNoExport{4}                   % all
\def\FfMultiline{4096}               % text
\def\FfPassword{8192}                % text
\def\FfNoToggleToOff{16384}          % radio
\def\FfRadio{32768}                  % radio
\def\FfPushButton{65536}             % Push button
\def\FfCombo{131072}                 % choice
\def\FfEdit{262144}                  % combo
\def\FfSort{524288}                  % choice
\def\FfFileSelect{1048576}           % text (PDF 1.4)
\def\FfMultiSelect{2097152}          % choice (PDF 1.4)
\def\FfDoNotSpellCheck{4194304}      % text, combo (PDF 1.4)
\def\FfDoNotScroll{8388608}          % text (PDF 1.4)
\def\FfComb{16777216}                % text (PDF 1.5)
\def\FfRadiosInUnison{33554432}      % radio (PDF 1.5)
\def\FfCommitOnSelChange{67108864}   % choice (PDF 1.5)
\def\FfRichText{33554432}            % radio (PDF 1.5)
%    \end{macrocode}
% The keys \texttt{/F} and \texttt{/Ff} will be additive, that is,
% for example, \verb+\F\FHidden\F\FPrint+ will get \texttt{/F 6},
% a field that is both printable and hidden. These are the only
% flags that are additive this way. The following to macros
% are supportive of the additivity.
%    \begin{macrocode}
\def\getFfValue/Ff#1\@nil{\def\eq@FfValue{#1}}
\def\getFValue/F#1\@nil{\def\eq@FValue{#1}}
\def\@getCmdName#1{\edef\@CmdName{\expandafter\@gobble\string#1}}
%    \end{macrocode}
%
% \subsection{The \texorpdfstring{\protect\cs{every\dots}}{\textbackslash{every...}} Commands}
%
%    \begin{macro}{\everyTextField}
%    \begin{macro}{\everySigField}
% Insert optional arguments for every \cs{textField}.
%    \begin{macrocode}
\newcommand{\everyTextField}[1]{\def\every@TextField{#1}}
\def\every@TextField{}
\newcommand{\everySigField}[1]{\def\every@sigField{#1}}
\def\every@sigField{}
%    \end{macrocode}
%    \end{macro}
%    \end{macro}
%    \begin{macro}{\everyCheckBox}
% Here, you can control the appearance of all the standard checkboxes, also
% effects radio fields of shortquiz and quiz.
%    \begin{macrocode}
\newcommand{\everyCheckBox}[1]{\def\every@CheckBox{#1}}
\def\every@CheckBox{}
%    \end{macrocode}
%    \end{macro}
%    \begin{macro}{\everyRadioButton}
% Pass key-values to every radio button through this key.
%    \begin{macrocode}
\newcommand{\everyRadioButton}[1]{\def\every@RadioButton{#1}}
\def\every@RadioButton{}
%    \end{macrocode}
%    \end{macro}
%    \begin{macro}{\everyButtonField}
%    \begin{macro}{\everyPushButton}
%    \begin{macro}{\everyListBox}
%    \begin{macro}{\everyComboBox}
%    \begin{macro}{\everyLink}
% Here, you can control the appearance of all the standard buttons.
%    \begin{macrocode}
\newcommand{\everyButtonField}[1]{\def\every@ButtonField{#1}}
\def\every@ButtonField{}
\newcommand{\everyPushButton}[1]{\def\every@PushButton{#1}}
\def\every@PushButton{}
\newcommand{\everyListBox}[1]{\def\every@listBox{#1}}
\newcommand{\everyComboBox}[1]{\def\every@comboBox{#1}}
\def\every@listBox{}\def\every@comboBox{}
\newcommand{\everyLink}[1]{\def\every@Link{#1}}
\def\every@Link{}
%    \end{macrocode}
%    \end{macro}
%    \end{macro}
%    \end{macro}
%    \end{macro}
%    \end{macro}
%
% \subsection{Set Field Properties}
%
% These are the two commands that process the field properties. This is
% done by calling \cs{processAppArgs} to loop through pairs of tokens
% and executing them, as explained in the discussion of \cs{processAppArgs}.
% \medskip\par\noindent
% The \DescribeMacro{\ef@djXPD}\cs{ef@djXPD} command attempts to reduce the size of a form field created by \app{xelatex} so its
% dimensions are consistent with what \app{pdflatex}/\app{Distiller} produce.
%    \begin{macrocode}
\def\ef@adjrectWH#1{\dimen@ii#1\relax
    \ifx\eq@rectW\@empty\else
        \eflength\eq@rectW\relax
        \advance\eflength\dimen@ii
        \edef\eq@rectW{\the\eflength}%
        \eflength\eq@rectH\relax
        \advance\eflength\dimen@ii
        \edef\eq@rectH{\the\eflength}%
    \fi
}
\def\ef@djXPD{\ifxetex\ifmakeXasPD\ef@adjrectWH{2bp}\fi
    \else\ifmakePDasX\ef@adjrectWH{-2bp}\fi\fi}
%    \end{macrocode}
%    \begin{macro}{\eq@setButtonProps}
% This macro measure the width of the largest text on defined for the
% button, and then passes the this info on to a driver specific macro
% \#1, the first parameter.
%\begin{verbatim}
% #1 is the driver specific macro to build the button widget
% #2 are the button properties
%\end{verbatim}
%\changes{v2.5m}{2012/01/27}{Added \string\cs{ef@btnspcr} to give user
% ability to adjust the spacing around text for a button.}
%    \begin{macrocode}
%\def\ef@btnspcr{\ }
\def\ef@btnspcr{}
\def\eq@setButtonProps#1#2{%
    \makeJSspecials
    \processAppArgs#2\end\@nil  % set widget properties
%    \end{macrocode}
%    Coordinate the values of \cs{BC} and \cs{W}, if one is empty
%    the other is too.
%    \changes{v2.9f}{2017/01/01}{BC=empty iff W=0 or empty}
%    \begin{macrocode}
    \ifx\eq@BC\@empty\@eqW{}\else
        \if\eq@W@value0\let\eq@BC\@empty\fi\fi
    \Hy@pdfstringfalse
    \ifx\eq@rectW\@empty
        \ifnum\eq@textSize=0 \else
%    \end{macrocode}
% If rectW is nonempty, and textSize is not zero, we calculate with
% width of the caption on the button by first adjusting the  for size
% to properly gauge the width of the text. This may not be really accurate
% because the font used by tex will no doubt be different from the font used
% by the button.
%    \begin{macrocode}
            \dimen@=\eq@textSize bp
            \dimen@1.00375\dimen@
            \edef\eq@textSize@pt{\strip@pt\dimen@}%
            \fontsize{\eq@textSize@pt}{0}\selectfont
        \fi
%    \end{macrocode}
% If the button is beveled, we pad the width by 2 times the width of the border,
% the beveled edge taking up a width approx equal to the border.
%    \begin{macrocode}
        \dimen@\eq@W@value bp
        \def\eq@S@B{B}\ifx\eq@S@value\eq@S@B
            \def\eq@btn@sp{\hbox to2\dimen@{\hfill}}%
        \else
            \def\eq@btn@sp{\hbox to\dimen@{\hfill}}%
        \fi
        \expandafter\def\expandafter
            \ef@btnspcr\expandafter{\ef@btnspcr\eq@btn@sp}%
        \sbox{\eq@tmpbox}{\ef@btnspcr\eq@CA\ef@btnspcr}%
           \eq@tmpdima=\wd\eq@tmpbox
        \sbox{\eq@tmpbox}{\ef@btnspcr\eq@RC\ef@btnspcr}%
        \ifdim\eq@tmpdima>\wd\eq@tmpbox\else
           \eq@tmpdima=\wd\eq@tmpbox\fi%
        \sbox{\eq@tmpbox}{\ef@btnspcr\eq@AC\ef@btnspcr}%
        \ifdim\eq@tmpdima>\wd\eq@tmpbox\else
            \eq@tmpdima=\wd\eq@tmpbox\fi
%    \end{macrocode}
%    (2017/01/22) If X-like, increase by 2bp
%    \changes{v2.9k}{2017/01/22}{If X-like, increase by 2bp}
%    \begin{macrocode}
        \ifmakePDasX\advance\eq@tmpdima2bp\fi
        \wd\eq@tmpbox=\eq@tmpdima
    \else
       \wd\eq@tmpbox=\eq@rectW
    \fi
%    \end{macrocode}
%    (2016/12/22) \cs{ef@djXPD} adjusts the size of the field dimensions, if \cs{makeXasPD} is true.
%    \begin{macrocode}
    \ef@djXPD#1% dps 12/22
}
%    \end{macrocode}
%    \end{macro}
%    \begin{macro}{\eq@setWidgetProps}
% Same as \cmd{\eq@setButtonProps} but does not measure the width of the
% field.  Simply lays in the optional parameters that modify the appearance
% then calls the driver specific macro to build the widget.
%\begin{verbatim}
% #1 is the driver specific macro to build the widget
% #2 are the widget properties
%\end{verbatim}
%    \begin{macrocode}
\def\eq@setWidgetProps#1#2{%
    \makeJSspecials
    \processAppArgs#2\end\@nil  % set widget properties
%    \end{macrocode}
%    Coordinate the values of \cs{BC} and \cs{W}, if one is empty
%    the other is too. This rule does not apply to links.
%    \changes{v2.9f}{2017/01/01}{BC=empty iff W=0 or empty}
%    \begin{macrocode}
    \ifx\annot@type@link\annot@type\else
        \ifx\eq@BC\@empty\@eqW{}\else
        \if\eq@W@value0\let\eq@BC\@empty\fi\fi
    \fi
%    \end{macrocode}
%    (2016/12/22) \cs{ef@djXPD} adjusts the size of the field dimensions, if \cs{makeXasPD} is true.
%    \begin{macrocode}
    \ef@djXPD#1% dps 12/22
}
%    \end{macrocode}
%    \end{macro}
% We now begin creating various commands for creating Acrobat Form fields. These
% fall into four categories: \hyperref[choice]{Choice Fields} (list box and the combo box),
% \hyperref[button]{Button Fields} (push button, check box, and radio button), \hyperref[textfield]{Text Fields},
% and \hyperref[sigfield]{Signature Fields}.
%
% Each of these fields has an optional first parameter which is used to
% change the appearance properties of the field, to set action of the
% field, and so on.  For this optional parameter, we sanitize certain
% characters so they can be used in, for example, urls, or JavaScript strings.
%\changes{v2.5k}{2011/01/19}
%{%
%   Added double quotes as other to the \string\cs{ef@sanitize@toks} command to
%   keep Babel from changing the JavaScript string. This definition remains
%   in effect only for the optional argument for form fields and links.
%}
%    \begin{macrocode}
\def\ef@sanitize@toks{\@makeother\~\@makeother\#\@makeother\&%
    \@makeother\"\@makeother\_}
%    \end{macrocode}
%    \begin{macro}{\efKern}
% A convenience macro to adjust spacing between fields.
%\begin{itemize}
%   \item \texttt{\#1} is the spacing for ps/pdftex drivers
%   \item \texttt{\#2} is the spacing for xetex
%\end{itemize}
%Unlike ps/pdftex, xetex includes the border width as part of the bounding rectangle
%dimensions. As a result, this wider boundary is known to {\TeX} as it lays out the line. For
% ps/pdftex, the boundary width is invisible to {\TeX}. The command \cs{ef@adjHWxetex} tries to
% adjust the dimensions for xetex so the form field dimensions are the over all platforms. The boundary
% is still ``visible'' in {\TeX} space.
%
%\paragraph*{Sample uses:}
%\begin{enumerate}
%   \item Two contiguous fields of the same border color \verb!\efKern{1bp}{-1bp}!. The right and
%         left borders exactly overlap.
%   \item Two contiguous fields of different border colors \verb!\efKern{2bp}{0bp}!. The right
% and left border are contiguous but do not overlap.
%\end{enumerate}
%    \begin{macrocode}
\newcommand\efKern[2]{\ifxetex\kern#2\else\kern#1\fi}
%    \end{macrocode}
%    \end{macro}
%    \DescribeMacro{\olBdry}\cs{olBdry} and \DescribeMacro{\cgBdry}\cs{cgBdry}
%    are convenience commands to setting the gap between two contiguous fields.
%    \cs{olBdry} gives positions the fields so that the boundary lines overlap
%    (ol) and \cs{cgBdry} positions the field so the boundary lines are
%    contiguous (cg).
%    \changes{v2.9d}{2016/12/22}{Added \string\cs{olBdry} and\string\cs{cgBdry}}
%    \begin{macrocode}
\newcommand\olBdry{\bgroup\ifxetex % dps 12/22
    \@tempdima-\g@eq@W@value@bp\relax
    \edef\@mtkern{\the\@tempdima}\else
    \@tempdima2bp\relax\advance\@tempdima-\g@eq@W@value@bp\relax % 12/26
    \edef\@mtkern{\the\@tempdima}\fi\kern\@mtkern\egroup
}
%    \end{macrocode}
%    (2017/01/15) Add optional argument to \cs{cgBdry}
%    \changes{v2.9h}{2017/01/15}{Add optional argument to \string\cs{cgBdry}}
%    \begin{macrocode}
\newcommand\cgBdry[1][0bp]{\bgroup\def\ef@rgi{#1}\ifx\ef@rgi\@empty
    \def\ef@rgi{0bp}\fi\setlength{\@tempdima}{\ef@rgi}%
    \ifxetex\else\addtolength{\@tempdima}{2bp}\fi
    \kern\@tempdima\egroup\ignorespaces}
%    \end{macrocode}
%    \DescribeMacro{\volBdry}\cs{volBdry} and \DescribeMacro{\vcgBdry}\cs{vcgBdry}
%    are convenience commands to setting the gap between two vertically oriented fields.
%    \cs{volBdry} gives positions the fields so that the boundary lines overlap
%    (ol) and \cs{vcgBdry} positions the field so the boundary lines are
%    contiguous (cg).
%    \changes{v2.9h}{2017/01/15}{Added \string\cs{volBdry} and\string\cs{vcgBdry}}
%    \changes{v2.9n}{2017/09/04}{Added \string\cs{efSupprIndent}}.
%    \begin{macrocode}
\newcommand{\efSupprIndent}{\ef@scratchtoks=\expandafter{\the\everypar}%
  \everypar{{\setbox\z@\lastbox}\clubpenalty\@M
  \everypar=\expandafter{\the\ef@scratchtoks}}}
\newcommand\volBdry{\bgroup\parskip0pt\relax\@@par\nointerlineskip
    \olBdry\egroup\efSupprIndent}
\newcommand\vcgBdry{\@ifstar{\edef\ef@offset{\the\parskip}\vcgBdry@i}
    {\def\ef@offset{0pt}\vcgBdry@i}}
\newcommand\vcgBdry@i[1][0bp]{\bgroup
    \setlength{\ef@dimena}{#1-\ef@offset}\parskip0pt\relax
    \par\nointerlineskip\cgBdry[\ef@dimena]%
    \egroup\ignorespaces\efSupprIndent}
%    \end{macrocode}
%
% \subsection{Choice Fields}\label{choice}
%
% This is the form template used for all choice fields, list box and combo box.
%    \begin{macrocode}
\def\common@choiceCode{%
    /Subtype/Widget
    /T (\Fld@name)
    /FT/Ch
    \eq@Ff
    \eq@F
    \eq@TU
    \eq@TI
    /BS << \eq@W\eq@S >>
    /MK <<\eq@R\eq@BC\eq@BG\eq@mkIns>>
    /DA (\eq@DA)
    /Opt [\eq@Opt]
    \eq@DV\eq@V
    \eq@A\eq@AA
    \eq@rawPDF
}
%    \end{macrocode}
% Sets the dimensions of the fields/links based on \texttt{\#1} (width)
% and \texttt{\#2} (height).
%    \begin{macrocode}
\def\eqf@setDimens#1#2{\@eqrectW{#1}\@eqrectH{#2}}
%    \end{macrocode}
% The \cs{ef@pdfstrCLOpt} command runs the option array for list and combo box through
% \cs{pdfstringdef}. There are two forms:
% \begin{enumerate}
%   \item |[(1)(Zur Info)][(2)(Bitte um R\"{u}cksprache)]|
%   \item |(Zur Info)(Bitte um R\"{u}cksprache)|
%\end{enumerate}
% In the code below, we distinguish these two cases based on the first character:
% if \texttt{[} then it is form (1), otherwise, it is form (2). The \cs{pdfstringdef}
% will convert the accented character \verb!\"{u}! into a PD1 character. (2015/06/06)
% the token \DescribeMacro\passthruCLOpts\cs{passthruCLOpts} is used to pass through
% raw option syntax for combo and list boxes; this allows a unicode notation.
%    \begin{macrocode}
\let\ef@@nil\relax
\def\@gobbleto@@nil#1\ef@@nil{}%
%    \end{macrocode}
% When \cs{passthruCLOpts} (or \texttt{*}) is used, there may be a
% lingering space that survives after the closing right brace of the
% argument, we attempt to absorb it with \cs{@gobbleto@@nil} immediately
% after the \cs{g@addto@macro}.
%    \begin{macrocode}
\long\def\g@addto@macrogobble#1#2{\g@addto@macro{#1}{#2}\@gobbleto@@nil}
%    \end{macrocode}
% \changes{v2.8a}{2015/07/15}{New def for \string\cs{passthruCLOpts}, and modified
% \cs{ef@pdfstrCLOpts} and its spawn.}
% (2015/07/15) New def for \cs{passthruCLOpts}, and modified \cs{ef@pdfstrCLOpts} and its spawn.
%    \begin{macrocode}
\def\passthruCLOpts{*}
\def\ef@pdfstrCLOpt{\Hy@unicodefalse\def\eq@Opt{}\ef@pdfstrCLOpti}
\def\ef@pdfstrCLOpti{\@ifnextchar\ef@@nil{\@gobble}{\ef@pdfstrCLOptia}}
%    \end{macrocode}
% The final argument of \cs{comboBox} or \cs{listBox} is any of three forms.
%\begin{verbatim}
%   {[(1)(Socks)][(2)(Shoes)][(3)(Pants)][(4)(Shirts)][(5)(Tie)]}
%   {(Socks)(Shoes)(Pants)(Shirts)(Tie)}
%   {\passthruCLOpts{%
%        [(Euro)<\unicodeStr(myEuro)>]%
%        [(Yen)<\unicodeStr(myYen)>]
%        [(Sheqel)<\unicodeStr(mySheqel)>]%
%        [(Pound)<\unicodeStr(myPound)>]%
%        [(Franc)<\unicodeStr(myFranc)>]}
%\end{verbatim}
% As long as the first token is not \cs{passthruCLOpts}, the argument may also be
% a combination of the first two, such as
%\begin{verbatim}
%   {[(1)(Socks)](Shoes)[(3)(Pants)](Shirts)[(5)(Tie)]}
%\end{verbatim}
% The  trick is to identify the first token of the group (\texttt{[} or \texttt{(}).
% When the first token is \cs{passthruCLOpts}, the extensive parse does not take place,
% and the whole argument just passes to \cs{eq@Opt}.
%\medskip\noindent
% If next char is `\texttt{[}' then we have an array element of the form
% \verb![(val)(appr)]!, we \cs{@gobble} the left brace in this case. If
% the next char is not a left bracket, go to next state.
%    \begin{macrocode}
\def\ef@pdfstrCLOptia{\@ifnextchar[{\expandafter
    \ef@pdfstrOptWBii\@gobble}{\ef@pdfstrCLOptb}}
%    \end{macrocode}
% If not a left bracket, check for \cs{passthruCLOpts}, which expands to `\texttt{*}'.
% If `\texttt{*}', then we pass the whole argument to \cs{eq@Opt} and end parsing. If
% not `\texttt{*}', check for parentheses.
%    \begin{macrocode}
\def\ef@pdfstrCLOptb{\@ifstar{\g@addto@macrogobble\eq@Opt}
    {\ef@pdfstrOptWPi}}
%    \end{macrocode}
%\changes{v2.6d}{2014/04/26}{Added additional parsing, so spaces
% can occur between arguments.}
% (2014/04/26) Added additional parsing, so spaces
% can occur between arguments.
%\par\medskip\noindent
% Following a bracket, the next token can only be a left parenthesis, if not error.
%    \begin{macrocode}
\def\ef@pdfstrOptWBii{%
    \@ifnextchar({\ef@pdfstrOptWBiia}{\PackageError{eforms}
        {Left parenthesis expected here}{}}%
}
%    \end{macrocode}
% We process the element of the form by pushing \verb![(val)(appr)]! onto the
% \cs{eq@Opt} stack.
%    \begin{macrocode}
\def\ef@pdfstrOptWBiia(#1){%
    \g@addto@macro\eq@Opt{[(}%
    \pdfstringdef\@optTokstemp{#1}%
    \expandafter\g@addto@macro\expandafter\eq@Opt
        \expandafter{\@optTokstemp}%
    \@ifnextchar({\ef@pdfstrOptWBiib}{\PackageError{eforms}
        {Left parenthesis expected here}{}}%
}
%    \end{macrocode}
% Get the second element of the array, which is enclosed in parentheses.
% push its value onto the stack too.
%    \begin{macrocode}
\def\ef@pdfstrOptWBiib(#1){%
    \g@addto@macro\eq@Opt{)(}%
    \pdfstringdef\@optTokstemp{#1}%
    \expandafter\g@addto@macro\expandafter\eq@Opt
        \expandafter{\@optTokstemp}%
    \g@addto@macro\eq@Opt{)]}%
%    \end{macrocode}
% Now after removing the final right bracket, return to the beginning,
% which is the \cs{ef@pdfstrCLOpti} command.
%    \begin{macrocode}
    \expandafter\ef@pdfstrCLOpti\@gobble
}
%    \end{macrocode}
% Process the purely parentheses version of the array.
%    \begin{macrocode}
\def\ef@pdfstrOptWPi{\@ifnextchar\ef@@nil{\@gobble}{\ef@pdfstrOptWPii}}
\def\ef@pdfstrOptWPii(#1){%
    \g@addto@macro\eq@Opt{(}%
    \pdfstringdef\@optTokstemp{#1}%
    \expandafter\g@addto@macro\expandafter\eq@Opt
        \expandafter{\@optTokstemp}%
    \g@addto@macro\eq@Opt{)}%
%    \end{macrocode}
% Return to the beginning, \cs{ef@pdfstrCLOpti}.
%    \begin{macrocode}
    \ef@pdfstrCLOpti}
%    \end{macrocode}
% \subsubsection{List Box}\label{listbox}
% The main list box code that can be used to build list box commands, such as
% \cs{listBox}, defined below.
%    \begin{macrocode}
\def\annot@type@listbox{listbox}
\newcommand\list@@Box[8]{\begingroup
    \edef\annot@type{\annot@type@listbox}%
    \pdfstringdef\Fld@name{#2}%
%    \edef\Fld@name{#2}%\def\eq@Opt{#5}%
%    \end{macrocode}
% Run \texttt{\#5} through \cs{ef@pdfstrCLOpt}.
%    \begin{macrocode}
    \expandafter\ef@pdfstrCLOpt#5\ef@@nil
    \eqf@setDimens{#3}{#4}%
    \def\eq@DA{\eq@textFont\space\eq@textSize\space Tf \eq@textColor}%
    \@processEvery#8\end\noindent#6#7{#1}%
}
%    \end{macrocode}
%    \begin{macro}{\listBox}
%\begin{verbatim}
% #1 = optional, used to enter any modification of the appearance/actions
% #2 = the title of the list box field
% #3 = the width of the bounding rectangle
% #4 = the height of the bounding rectangle
% #5 = the face values/export values of list.
%\end{verbatim}
%    \begin{macrocode}
\def\listBoxDefaults{%
    \W{1}\S{I}\F{\FPrint}\BC{0 0 0}
}
%    \end{macrocode}
% We sanitize the optional first parameter.
%    \begin{macrocode}
\newcommand\listBox{\begingroup
    \ef@sanitize@toks\ef@listbox
}
\newcommand{\ef@listbox}[1][]{%
    \endgroup\ef@listBox[#1]%
}
\newcommand\ef@listBox[5][]{%
    \mbox{\list@@Box{#1}{#2}{#3}{#4}{#5}{}{\eq@setWidgetProps
        \eq@choice@driver}{\listBoxDefaults\every@listBox}}%
}
%    \end{macrocode}
%    \end{macro}
% \subsubsection{Combo Box}\label{combobox}
%    \begin{macrocode}
\def\annot@type@combobox{combobox}
\newcommand\combo@@Box[8]{\begingroup
   \edef\annot@type{\annot@type@combobox}%
    \@eqFf{\FfCombo}\pdfstringdef\Fld@name{#2}%
%    \edef\Fld@name{#2}%\def\eq@Opt{#5}%
%    \end{macrocode}
% Run \texttt{\#5} through \cs{ef@pdfstrCLOpt}.
%    \begin{macrocode}
    \expandafter\ef@pdfstrCLOpt#5\ef@@nil
    \eqf@setDimens{#3}{#4}%
    \def\eq@DA{\eq@textFont\space\eq@textSize\space Tf \eq@textColor}%
    \@processEvery#8\end\noindent#6#7{#1}%
}
%    \end{macrocode}
%    \begin{macro}{\comboBox}
% A general combo box command.
%\begin{verbatim}
% #1 = optional, used to enter any modification of the appearance/actions
% #2 = the title of the radio field
% #3 = the width of the bounding rectangle
% #4 = the height of the bounding rectangle
% #5 = the face values/export values of combo box.
%\end{verbatim}
%    \begin{macrocode}
\def\comboBoxDefaults{%
    \W{1}\S{I}\F{\FPrint}\BC{0 0 0}
}
%    \end{macrocode}
% We sanitize the optional first parameter.
%    \begin{macrocode}
\newcommand\comboBox{\begingroup
    \ef@sanitize@toks\ef@combobox
}
\newcommand{\ef@combobox}[1][]{%
    \endgroup\ef@comboBox[#1]%
}
\newcommand\ef@comboBox[5][]{%
    \mbox{\combo@@Box{#1}{#2}{#3}{#4}{#5}{}{\eq@setWidgetProps
        \eq@choice@driver}{\comboBoxDefaults\every@comboBox}}%
}
%    \end{macrocode}
%    \end{macro}
% \subsection{Button Fields}\label{button}
% Here is the field template for push button fields.
%    \begin{macrocode}
\def\common@pushButtonCode{%
    /Subtype/Widget
    /T (\Fld@name)
    /FT/Btn
    \eq@Ff
    \eq@TU
    \eq@H
    \eq@F
    /BS <<\eq@W\eq@S >>
\ifx\eq@AP\@empty
    /MK <<\eq@R\eq@BC\eq@BG%
         \ef@kvCA\ef@kvRC\ef@kvAC\eq@IconMK\eq@mkIns>>
\else
    \eq@AP
\fi
    /DA (\eq@DA)
    \eq@A\eq@AA
    \eq@rawPDF
}
%    \end{macrocode}
% Here is the field template for check boxes and radio button fields fields.
%    \begin{macrocode}
\def\common@RadioCheckCode{%
    /Subtype/Widget
    /T (\Fld@name)
    /FT/Btn
    \eq@Ff
    \eq@F
    \eq@TU
    /BS <<\eq@W\eq@S>>
%    \end{macrocode}
% \changes{v1.0d}{2007/09/01}{%
%    Corrected a problem with radio buttons. The problem was created by earlier
%    changes so that the \string\textbf{AP} key could play a greater role in creating
%    different appearances.
%}
%    \begin{macrocode}
\ifx\eq@AP\@empty
    /AP<< /N <<\eq@On<<>>>> >>
    \eq@MK
\else
    \eq@AP
\fi
    /DA (\eq@DA)
    \eq@AS
    \eq@DV\eq@V
    \eq@A\eq@AA
    \eq@rawPDF
}
%    \end{macrocode}
% \subsubsection{Push Button}\label{pushbutton}
% Here is the basic command for creating a button field. This is the building block
% for all other buttons.
%\changes{v2.0b}{2008/06/19}
%{
%   Added code to take the dimensions and pass them through
%   \string\cs{setlength}, so that calc type dimensions can be used
%   when setting the dimensions of this widget. Made the same
%   changes for all AcroForm elements and for linking.
%}
%    \begin{macrocode}
\def\annot@type@button{pushbtn}
\newcommand\push@@Button[7]{\begingroup
    \edef\annot@type{\annot@type@button}%
    \pdfstringdef\Fld@name{#2}%\edef\Fld@name{#2}%
    \makeJSspecials\ef@preProcDefns
    \def\eq@Ff{/Ff \FfPushButton}%
    \def\eq@DA{\eq@textFont\space\eq@textSize\space Tf \eq@textColor}%
    \eqf@setDimens{#3}{#4}%
    \@processEvery#7\end\noindent#5#6{#1}%
}
%    \end{macrocode}
%    \begin{macro}{\pushButton}
%\begin{verbatim}
%#1 = optional, used to enter any modification of the appearance/actions
%#2 = the title of the button field
%#3 = the width of the bounding rectangle
%#4 = the height of the bounding rectangle
%\end{verbatim}
%    \begin{macrocode}
\def\pushButtonDefaults{%
    \W{1}\S{B}\F{\FPrint}\BC{0 0 0}
    \H{P}\BG{.7529 .7529 .7529}
}
%    \end{macrocode}
% We sanitize the optional first parameter.
%    \begin{macrocode}
\newcommand\pushButton{\begingroup
    \ef@sanitize@toks\ef@pushbutton
}
\newcommand{\ef@pushbutton}[1][]{%
    \endgroup\ef@pushButton[#1]%
}
\newcommand\ef@pushButton[4][]{%
    \mbox{\push@@Button{#1}{#2}{#3}{#4}{}{%
        \eq@setButtonProps\eq@Button@driver}%
        {\pushButtonDefaults\every@PushButton}}%
}
%    \end{macrocode}
%    \end{macro}
% \subsubsection{Check Box}\label{checkbox}
% The basic command for creating check boxes.
%    \begin{macrocode}
\def\annot@type@checkbox{checkbox}
\newcommand\check@@Box[8]
{%
    \begingroup\let\#\ef@Hx
    \edef\annot@type{\annot@type@checkbox}%
    \pdfstringdef\Fld@name{#2}\@eqAS{Off}%
    \def\@eqDV##1{\def\eq@arg{##1}\ifx\eq@arg\@empty\let\eq@DV\@empty
        \else\ifpdfmarkup\def\eq@DV{/DV(##1) cvn }\else
        \def\eq@DV{/DV/##1}\fi\fi}%
    \def\@eqV##1{\def\eq@arg{##1}\ifx\eq@arg\@empty
        \let\eq@V\@empty\else\ifpdfmarkup\def\eq@V{/V(##1) cvn }\else
        \def\eq@V{/V/##1}\fi\@eqAS{##1}\fi}%
    \eqf@setDimens{#3}{#4}%
    \ifpdfmarkup\def\eq@On{(#5) cvn }\else
        \def\eq@On{/#5}\fi\@eqtextFont{ZaDb}%
    \def\eq@DA{\eq@textFont\space\eq@textSize\space Tf \eq@textColor}%
    \@eqMK{\eq@R\eq@BC\eq@BG/CA(\symbol@choice)\eq@mkIns}%
    \@processEvery#8\end\noindent#6#7{#1}%
}
%    \end{macrocode}
%    \begin{macro}{\checkBox}
%\begin{verbatim}
%#1 = optional, used to enter any modification of the appearance/actions
%#2 = the title of the check box field
%#3 = the width of the bounding rectangle
%#4 = the height of the bounding rectangle
%#5 = the 'on value' or export value, the default is "Yes".
%\end{verbatim}
%    \begin{macrocode}
\def\checkBoxDefaults{%
    \F{\FPrint}\W{1}\S{S}\BC{0 0 0}%
}
\newcommand\checkBox{\begingroup
    \ef@sanitize@toks\ef@checkbox
}
\newcommand{\ef@checkbox}[5][]{%
    \endgroup\mbox{\check@@Box{#1}{#2}{#3}{#4}{#5}{}{\eq@setWidgetProps
    \eq@RadioCheck@driver}{\checkBoxDefaults\every@CheckBox}}%
}
%    \end{macrocode}
%    \end{macro}
% \subsubsection{Radio Button}\label{radiobutton}
% The basic command for creating radio button fields.
%    \begin{macrocode}
% Basic command for building all radio buttons.
%    \end{macrocode}
%    \begin{macrocode}
\def\annot@type@radio{radiobtn}
\newcommand\radio@@Button[8]{\begingroup\let\#\ef@Hx
    \edef\annot@type{\annot@type@radio}%
    \pdfstringdef\Fld@name{#2}\@eqAS{Off}%
%    \edef\Fld@name{#2}\@eqAS{Off}%
    \def\@eqDV##1{\def\eq@arg{##1}\ifx\eq@arg\@empty\let\eq@DV\@empty
        \else\ifpdfmarkup\def\eq@DV{/DV(##1) cvn }\else
        \def\eq@DV{/DV/##1}\fi\fi}%
    \def\@eqV##1{\def\eq@arg{##1}\ifx\eq@arg\@empty
        \let\eq@V\@empty\else\ifpdfmarkup\def\eq@V{/V(##1) cvn }\else
        \def\eq@V{/V/##1}\fi\@eqAS{##1}\fi}%
    \eqf@setDimens{#3}{#4}%
    \ifpdfmarkup\def\eq@On{(#5) cvn }\else\def\eq@On{/#5}\fi
    \def\eq@Ff{/Ff \FfRadio}\@eqtextFont{ZaDb}%
    \def\eq@DA{\eq@textFont\space\eq@textSize\space Tf \eq@textColor}%
    \@eqMK{\eq@R\eq@BC\eq@BG/CA(\symbol@choice)\eq@mkIns}%
    \@processEvery#8\end\noindent#6#7{#1}%
}
%    \end{macrocode}
%    \begin{macro}{\radioButton}
%\begin{verbatim}
%#1 = optional, used to enter any modification of the appearance/actions
%#2 = the title of the radio button field
%#3 = the width of the bounding rectangle
%#4 = the height of the bounding rectangle
%#5 = the 'on value' or export value, the default is "Yes"
%\end{verbatim}
%    \begin{macrocode}
\def\radioButtonDefaults
{%
    \W{1}\S{S}\BC{0 0 0}\F{\FPrint}
}
%    \end{macrocode}
% We sanitize the optional first parameter.
%    \begin{macrocode}
\newcommand\radioButton{\begingroup
    \ef@sanitize@toks\ef@radiobutton
}
\newcommand{\ef@radiobutton}[5][]{\endgroup
    \mbox{\radio@@Button{#1}{#2}{#3}{#4}{#5}{}{\eq@setWidgetProps
    \eq@RadioCheck@driver}{\radioButtonDefaults\every@RadioButton}}%
}
\newcommand\ef@radioButton[5][]
{%
    \mbox{\radio@@Button{#1}{#2}{#3}{#4}{#5}{}{\eq@setWidgetProps
        \eq@RadioCheck@driver}{\radioButtonDefaults\every@RadioButton}}%
}
%    \end{macrocode}
%    \end{macro}
%
% \subsection{Text Field}\label{textfield}
%
% The template for a text field.
%    \begin{macrocode}
\def\common@TextFieldCode
{%
    /Subtype/Widget
    /T (\Fld@name)
    /FT/Tx
    \eq@Ff
    \eq@F
    \eq@Q
    \eq@TU
    \eq@MaxLen
    /BS <<\eq@W\eq@S>>
    /MK <<\eq@R\eq@BC\eq@BG\eq@mkIns>>
    /DA (\eq@DA)
    \eq@DV\eq@V
    \eq@RV\eq@DS
    \eq@A\eq@AA
    \eq@rawPDF
}
%    \end{macrocode}
% The basic text field macro for constructing all other text fields.
%    \begin{macrocode}
\def\annot@type@text{textfld}
\newcommand\text@@Field[7]
{%
    \begingroup
    \edef\annot@type{\annot@type@text}%
    \pdfstringdef\Fld@name{#2}\def\eq@Title{#2}%
%    \edef\Fld@name{#2}\def\eq@Title{#2}%
    \eqf@setDimens{#3}{#4}%
    \def\eq@DA{\eq@textFont\space\eq@textSize\space Tf \eq@textColor}%
    \@processEvery#7\end\noindent#5#6{#1}%
}
%    \end{macrocode}
%    \begin{macro}{\textField}
%\begin{verbatim}
%#1 = optional, used to enter any modification of the appearance/actions
%#2 = the title of the text field
%#3 = the width of the bounding rectangle
%#4 = the height of the bounding rectangle
%\end{verbatim}
%    \begin{macrocode}
\def\textFieldDefaults
{%
    \F{\FPrint}\BC{0 0 0}\W{1}\S{S}
}
%    \end{macrocode}
% We sanitize the optional first parameter.
%    \begin{macrocode}
\newcommand\textField{\begingroup
    \ef@sanitize@toks\ef@textfield
}
\newcommand{\ef@textfield}[2][]{%
    \endgroup\ef@textField[#1]{#2}%
}
\newcommand\ef@textField[4][]
{%
    \mbox{\text@@Field{#1}{#2}{#3}{#4}{}%
        {\eq@setWidgetProps\eq@TextField}%
        {\textFieldDefaults\every@TextField}}%
}
%    \end{macrocode}
% Some legacy assignments.
%    \begin{macrocode}
\let\eqTextField\textField
\let\calcTextField\textField
%    \end{macrocode}
%    \end{macro}
%
% \subsection{Signature Field}\label{sigfield}
%
% The template for a signature field.
%    \begin{macrocode}
\def\common@SigFieldCode
{%
    /Subtype /Widget
    /T (\Fld@name)
    /FT/Sig
    \eq@F
    \eq@TU
    /BS <<\eq@W\eq@S>>
    /MK <<\eq@R\eq@BC\eq@BG\eq@mkIns>>
    /DA (\eq@DA)
    \eq@Lock
    \eq@A\eq@AA
    \eq@rawPDF
}
%    \end{macrocode}
% The basic text field macro for constructing all other text fields.
%    \begin{macrocode}
\def\annot@type@sig{sigfld}
\newcommand\sig@@Field[7]{%
    \begingroup
    \edef\annot@type{\annot@type@sig}%
    \pdfstringdef\Fld@name{#2}\def\eq@Title{#2}%
%    \edef\Fld@name{#2}\def\eq@Title{#2}%
    \eqf@setDimens{#3}{#4}%
    \def\eq@DA{\eq@textFont\space\eq@textSize\space Tf \eq@textColor}%
    \@processEvery#7\end\noindent#5#6{#1}%
}
%    \end{macrocode}
%    \begin{macro}{\sigField}
%\begin{verbatim}
%#1 = optional, used to enter any modification of the appearance/actions
%#2 = the title of the text field
%#3 = the width of the bounding rectangle
%#4 = the height of the bounding rectangle
%\end{verbatim}
%    \begin{macrocode}
\def\sigFieldDefaults
{%
    \F{\FPrint}\BC{}\BG{}\W{1}\S{S}
}
%    \end{macrocode}
% We sanitize the optional first parameter.
%    \begin{macrocode}
\newcommand\sigField{\begingroup
    \ef@sanitize@toks\ef@sigfield
}
\newcommand{\ef@sigfield}[1][]{%
    \endgroup\ef@sigField[#1]%
}
\newcommand\ef@sigField[4][]
{%
    \mbox{\sig@@Field{#1}{#2}{#3}{#4}{}{\eq@setWidgetProps\eq@SigField}%
        {\sigFieldDefaults\every@sigField}}%
}
%    \end{macrocode}
%    \end{macro}
%
% \section{Additional Link Support}
%
% The links in \textsf{hyperref} are not sufficiently general to
% allow actions other than jumping. I've included a general link
% that increases the usage of the links provided by
% \textsf{hyperref}.
% \paragraph{Common Link Code.} All the link commands eventually come back
% to this template for constructing a link annotation.
%    \begin{macrocode}
\def\common@LinkCode
{%
    \eq@A           % Action
    \eq@H           % Highlight
    \eq@Color       % Border color
    \link@BS        % Border styles
    \eq@rawPDF      % everything else
}
%    \end{macrocode}
% The low-level link command and is the building block of all the
% other, more user-friendly link commands. Takes seven parameters:
%\begin{enumerate}
%\item[\texttt{\#1}:] Optional arguments to modify the appearance and actions of the link.
%\item[\texttt{\#2}:] The width of the bounding rectangle.
%\item[\texttt{\#3}:] The height of the bounding rectangle.
%\item[\texttt{\#4}:] The text to be enclosed by the link.
%\item[\texttt{\#5}:] Used to set link properties and link driver.
%\item[\texttt{\#6}:] I don't remember what this one was about, same as \texttt{\#5}.
%\item[\texttt{\#7}:] Used to set link defaults and to execute \cs{everyLink}.
%\end{enumerate}
%    \begin{macrocode}
\def\annot@type@link{link}
\newcommand\set@@Link[7]
{%
    \begingroup
    \makeJSspecials
    \edef\annot@type{\annot@type@link}%
    \ef@preProcDefns
    \eqf@setDimens{#2}{#3}%
    \ifx\eq@rectW\@empty\def\link@@Box{#4}\else
        \def\eq@arg{#4}\ifx\eq@arg\@empty
            \def\eq@content{\hfill\vfill}\else\def\eq@content{#4}\fi
        \def\link@@Box{\parbox[\eq@pos][\eq@rectH][\eq@innerpos]%
            {\eq@rectW}{\centering\eq@content}}%
    \fi
    \@processEvery#7\end\noindent#5#6{#1}%
}
%    \end{macrocode}
% \paragraph{The Visibility of Border of the Link.} We use \textsf{hyperref}'s
% option \texttt{colorlinks} to determine if we
% make the bounding rectangle visible or not by default.
%    \begin{macrocode}
\def\defaultlinkcolor{\@linkcolor}
%    \end{macrocode}
% Delay default link color until beginning of document.
%\changes{v2.8f}{2016/04/05}{Delay default link color until beginning of document.}
%    \begin{macrocode}
\def\setDef@ultLinkColor{\ifHy@colorlinks
    \def\ef@thislinkcolor{\defaultlinkcolor}
    \def\ef@colorthislink{\color{\ef@thislinkcolor}}\else
    \let\ef@colorthislink\relax\fi
}
\AtBeginDocument{\setDef@ultLinkColor}
%    \end{macrocode}
%    \begin{macro}{\setLink}
%    \begin{macro}{\mlsetLink}
%    \begin{macro}{\setLinkText}
% This is the basic high-level link command, it is used to surround text.
%\begin{verbatim}
% \setLinkText[\A{\JS{this.pageNum=6;}}\Color{1 0 0}]{Go There!}
%\end{verbatim}
% \paragraph{Link Parameters.} This link takes two parameters, one of which
% is optional.
%\begin{enumerate}
%\item[\texttt{[\#1]}:] Optional key-value pairs to change the appearance or action
% of this link.
%\item[\texttt{\#2}:] The text to be enclosed in the bounding rectangle of this link.
%\end{enumerate}
%\paragraph{Link default parameters.} We set the line style on solid, and
% the border array to \texttt{0 0 0}, no visible border.
%    \begin{macrocode}
\def\set@LinkTextDefaults{\S{S}\Border{0 0 0}}
%    \end{macrocode}
% If the \texttt{colorlinks} option is in effect with hyperref, we color links when author
% uses |\setLinkText|, a command used by many other commands defined in AeB.
%    \begin{macrocode}
\newcommand\setLink{\begingroup
    \ef@sanitize@toks\ef@setlinktext
}
\newcommand{\mlsetLink}{\mlhypertext}
\let\setLinkText\setLink
\newcommand{\ef@setlinktext}[1][]{%
%    \end{macrocode}
% \changes{v2.5b}{2009/12/24}{added path to \string\cs{mlhypertext}}
% Added a path to \cs{mlhypertext} of the \textsf{aeb\_mlink} package. When the
% user specifies \verb!\mlLink{true}! in the option list, we branch off to
% \cs{mlhypertext}.
%    \begin{macrocode}
    \endgroup\ef@searchmlLink#1\mlLink\end\@nil
    \ifx\ef@mlLink0\def\ef@next{\set@LinkText[#1]}\else
    \def\ef@next{\mlhypertext[#1]}\fi\ef@next
}
\newcommand\set@LinkText[2][]{%
    \set@@Link{#1}{}{}{\ef@colorthislink#2}{}%
        {\eq@setWidgetProps{\ef@postProcLinkProps\setLink@driver}}%
        {\set@LinkTextDefaults\every@Link}%
}
%    \end{macrocode}
% Definitions we want make locally to the options parameters; otherwise,
% these are undefined.
%    \begin{macrocode}
\def\ef@preProcDefns{%
    \def\Win##1{/Win <<##1>>}%
    \def\fitpage{\dl@fitpage}\def\actualsize{\dl@actualsize}%
    \def\fitwidth{\dl@fitwidth}\def\fitheight{\dl@fitheight}%
    \def\fitvisible{\dl@fitvisible}\def\inheritzoom{\dl@inheritzoom}%
    \let\rPage\ef@rPage
    \edef\Page##1{\ifcase\eq@drivernum
        {Page##1}\or
        \noexpand\pdfpageref##1\space\space 0 R\or
        \noexpand @page##1\fi
    }%
}
%    \end{macrocode}
% After the properties are processed, the flow comes to |\ef@postProcLinkProps|
% for one last chance to process the properties more finely. The flow continues
% to |\setLink@@driver|.
%    \begin{macrocode}
\def\ef@postProcLinkProps{}
%    \end{macrocode}
%    \end{macro}
%    \end{macro}
%    \end{macro}
%    \begin{macro}{\setLinkBbox}
% Set a link around a |\parbox|.
%\begin{verbatim}
%\setLinkBbox [\Color{1 0 0}}]{50bp}{30bp}[b]{\centering Press Me!}
%\end{verbatim}
% \paragraph{Link Parameters.} There are four parameters, two of which
% are optional.
%\begin{enumerate}
%\item[\texttt{[\#1]}:] Optional key-value pairs to change the appearance or action
% of this link.
%\item[\texttt{\#2}:] The width of the \cs{parbox}.
%\item[\texttt{\#3}:] The height of the \cs{parbox}.
%\item[\texttt{[\#4]}:] The positioning parameter of the \cs{parbox} (\texttt{b}, \texttt{c}, \texttt{t}).
%\item[\texttt{[\#5]}:] The text or object to be enclosed in a \cs{parbox}
%\end{enumerate}
%\paragraph{Link default parameters.} We set the line style on solid, and
% the border array to \texttt{0 0 0}, no visible border.
%    \begin{macrocode}
\def\set@LinkBboxDefaults{\S{S}\Border{0 0 0}}
%    \end{macrocode}
%    \begin{macrocode}
\newcommand{\setLinkBbox}{\begingroup
    \ef@sanitize@toks\ef@linkbbox
}
\newcommand{\ef@linkbbox}[1][]{%
    \endgroup\ef@setLinkBbox[#1]%
}
\newcommand{\ef@setLinkBbox}[3][]{%
    \@setLinkBbox{#1}{#2}{#3}%
}
\def\@setLinkBbox#1#2#3{\@ifnextchar[{\@@setLinkBbox{#1}{#2}{#3}}%
    {\@@setLinkBbox{#1}{#2}{#3}[c]}}
\def\@@setLinkBbox#1#2#3[#4]{%
    \@ifnextchar[{\@@@setLinkBbox{#1}{#2}{#3}{#4}}%
        {\@@@setLinkBbox{#1}{#2}{#3}{#4}[#4]}%
}
\def\@@@setLinkBbox#1#2#3#4[#5]#6{%
    \def\eq@pos{#4}\def\eq@innerpos{#5}%
    \set@@Link{#1}{#2}{#3}{\ef@colorthislink#6}%
        {\eq@setWidgetProps\setLink@driver}{}%
        {\set@LinkBboxDefaults\every@Link}%
}
%    \end{macrocode}
%    \end{macro}
%    \begin{macro}{\setLinkPbox}
% This is a special link for creating a fixed box around the
% current paragraph and is used with the multi-line link commands.
%\changes{v1.0c}{2007/07/21}
%{
%  Added \string\cs{setLinkPbox} to support multi-line links for dvips
% and dvipsone. Added to \string\cs{setLinkPbox@driver} a special link
% driver that supports \string\textbf{QuadPoints}.
%}
%    \begin{macrocode}
\def\set@LinkPboxDefaults{\S{S}\Border{0 0 0}}
\newcommand\setLinkPbox{\begingroup
    \ef@sanitize@toks\ef@linkpbox
}
\newcommand{\ef@linkpbox}[1]{%
    \endgroup\ef@setLinkPbox{#1}%
}
\newcommand\ef@setLinkPbox[1]{%
    \@setLinkPbox{#1\BS{}}{}{}{\hfill\vfill}%
}
\def\@setLinkPbox#1#2#3{\@ifnextchar[{\@@setLinkPbox{#1}{#2}{#3}}%
    {\@@setLinkPbox{#1}{#2}{#3}[c]}}
\def\@@setLinkPbox#1#2#3[#4]{%
    \@ifnextchar[{\@@@setLinkPbox{#1}{#2}{#3}{#4}}%
        {\@@@setLinkPbox{#1}{#2}{#3}{#4}[#4]}
}
\def\@@@setLinkPbox#1#2#3#4[#5]#6{%
    \def\eq@pos{#4}\def\eq@innerpos{#5}%
    \set@@Link{#1}{#2}{#3}{#6}{\eq@setWidgetProps\setLinkPbox@driver}%
        {}{\set@LinkPboxDefaults\every@Link}%
}
%    \end{macrocode}
%    \end{macro}
%    \begin{macrocode}
%</package>
%    \end{macrocode}
%
% \section{A User Interface to links and forms}
%
% Through the years, I've developed my own version of a ``key-value''
% system, and this system works very well (see
% Section~\ref{procArgs}, page~\pageref{procArgs}, for the main
% macro); however, in a feeble attempt to make the system more user
% friendly, I've incorporated an \textsf{xkeyval} system into the
% \textsf{eforms} package. My old key-value system---which requires less
% typing and, in my opinion, is easier to use---still works,
% and the new one is just a user interface to the old system; parameters are
% passed to the main processing macro, defined in Section~\ref{procArgs}.
%    \begin{macrocode}
%<*userinterface>
%    \end{macrocode}
% Let's bring in the \textsf{xkeyval} package.
%    \begin{macrocode}
%\usepackage{xkeyval}
%    \end{macrocode}
% The |\ui| interface is undefined if the \texttt{useui} option is not taken. We use this fact
% to send a message to the user that using |\ui| most accompany the \texttt{useui}.
%    \begin{macrocode}
\def\@equi#1{}
%    \end{macrocode}
% The visibility of the border of a link depends on where the \texttt{colorlinks} option of hyperref has been
% taken. If the author does not take \texttt{colorlinks}, a visible rectangle appears by default. By putting
% \texttt{border=visible}, the author can override this behavior. If \texttt{colorlinks} option was taken, then the same
% link has an invisible border by default. Again, the author can override this by putting \texttt{border=visible}.
%    \begin{macrocode}
\ifHy@colorlinks\def\eq@bordervisibledefault{0}\@eqW{}\else
\def\eq@bordervisibledefault{1}\@eqW{1}\fi
%    \end{macrocode}
% \subsection{Some Utility Commands}
% Some utility commands that are defined elsewhere in the AeB, but this package does not assume AeB, so
% we'll define them here as well.
%    \begin{macrocode}
\def\getargs#1#2{\def\aeb@argi{#1}\def\aeb@argii{#2}}
\def\labelRef#1{\@ifundefined{r@#1}{Doc-Start}
    {\aeb@exiii\@fourthoffive\csname r@#1\endcsname}}
\def\aeb@exiii{\expandafter\expandafter\expandafter}
\newtoks\ef@flagtoks
\newtoks\ef@jstoks
\def\noexpandiii{\noexpand\noexpand\noexpand}
%    \end{macrocode}
%
% \subsection{The Appearance Tab}
%
% We set this user interface up to model the UI of Acrobat/Reader.
%    \IndexKey{border}
% In the case of a link, this is the Link Type: \textsf{Visible Rectangle}
% or \textsf{Invisible Rectangle}. For forms, this key has not counterpart
% in the user interface. If you set border equal to \texttt{invisible}, that
% will set border line width to zero |\W{0}|.
%    \begin{macrocode}
\define@choicekey{eforms}{border}[\val\nr]{visible,invisible}
{%
    \ifcase\nr
%    \end{macrocode}
% Author has chosen a visible border |\eq@W@buffered|.
%    \begin{macrocode}
        \def\eq@visibleborder{1}%
        \ifx\eq@W\@empty
            \ifx\eq@W@buffered\@empty\@eqW{1}\else
            \@eqW{\eq@W@buffered}\fi
        \else
            \ifnum\eq@W@value=0 \@eqW{1}\fi
            \ifx\eq@W@buffered\@empty\else
            \@eqW{\eq@W@buffered}\fi
        \fi
        \@eqBorder{0 0 \eq@W@value}%
    \or
        % author has chosen invisible border
        \def\eq@visibleborder{0}\@eqW{}%
        \@eqBorder{0 0 0}\@eqBC{}%
    \fi
}
%    \end{macrocode}
% This choice key is for the \texttt{linewidth}\IndexKey{linewidth} of a link or form. The user interface choices
% are \texttt{thin}, \texttt{medium}, and \texttt{thick}. This number is ignored if the document
% author has set the border to \texttt{invisible}.
%    \begin{macrocode}
\define@choicekey{eforms}{linewidth}[\val\nr]{thin,medium,thick}
{%
    \ifx\annot@type\annot@type@link
%    \end{macrocode}
% This is a link, so, depending on whether the color package is loaded, we show or don't
% show the boundary, depending on the explicate choices of the author
%    \begin{macrocode}
        \ifx\eq@visibleborder\@empty
%    \end{macrocode}
% Author has not committed to a border type, so we'll assume the default
%    \begin{macrocode}
            \ifnum\eq@bordervisibledefault=0
%    \end{macrocode}
% A color package is loaded, use invisible border by default.
%    \begin{macrocode}
                \edef\eq@W@buffered{\ifcase\nr 1\or2\or3\fi}%
                \@eqW{}%
            \else
%    \end{macrocode}
% No color is loaded, show the border.
%    \begin{macrocode}
                \ifcase\nr \@eqW{1}\or\@eqW{2}\or
                \@eqW{3}\else\@eqW{1}\fi\@eqBorder{0 0 \eq@W@value}%
            \fi
        \else
%    \end{macrocode}
% Author has set the border type before line width, we'll do what the
% author wants. If invisible, ignore |\W|, if visible obey |\W|.
%    \begin{macrocode}
            \ifnum\eq@visibleborder=1 % visible border
                \ifcase\nr \@eqW{1}\or\@eqW{2}\or
                \@eqW{3}\else\@eqW{1}\fi\@eqBorder{0 0 \eq@W@value}%
            \fi
        \fi
    \else
%    \end{macrocode}
% This is not a link, its a form, so things are easier. The invisible border will override
% the linewidth. If the author wants a border, s/he better get her/his act together and
% remove \texttt{border=invisible}.
%    \begin{macrocode}
        \ifx\eq@visibleborder\@empty
%    \end{macrocode}
% If the value of |\eq@visibleborder|, author has not used the border key at this time,
% so just set the \texttt{linewidth} as stated.
%    \begin{macrocode}
            \ifcase\nr \@eqW{1}\or\@eqW{2}\or
            \@eqW{3}\else\@eqW{1}\fi\@eqBorder{0 0 \eq@W@value}%
            \edef\eq@W@buffered{\ifcase\nr 1\or2\or3\fi}%
        \else
%    \end{macrocode}
% The author has used the border key, so we don't need to save this value.
%    \begin{macrocode}
            \ifnum\eq@visibleborder>0
                \ifcase\nr \@eqW{1}\or\@eqW{2}\or
                \@eqW{3}\else\@eqW{1}\fi\@eqBorder{0 0 \eq@W@value}%
            \fi
        \fi
    \fi
}
\let\eq@visibleborder\@empty
\let\eq@W@buffered\@empty
%    \end{macrocode}
%
% The highlight\IndexKey{highlight} type, choices are \texttt{none}, \texttt{invert},
% \texttt{outline}, \texttt{inset} and \texttt{push}. The term \texttt{inset}
% is used with links, and \texttt{push} is used with forms. They each have the
% same key value pair.
%    \begin{macrocode}
\define@choicekey{eforms}{highlight}[\val\nr]{none,invert,outline,%
    inset,push}
{%
    \ifcase\nr
      \@eqH{}\or\@eqH{I}\or\@eqH{O}\or\@eqH{P}\or\@eqH{P}\fi
}{}
%    \end{macrocode}
%    \IndexKey{bordercolor}
% This is an rgb color, \texttt{bordercolor=1 0 0}, is the color red.
%    \begin{macrocode}
\define@key{eforms}{bordercolor}[]{%
    \ifx\annot@type\annot@type@link\@eqColor{#1}\else
% If border is invisible, we ignore bordercolor
    \ifx\eq@visibleborder\@empty\@eqBC{#1}\else
        \ifnum\eq@visibleborder>0\relax\@eqBC{#1}\fi\fi\fi
}
%    \end{macrocode}
%
% The line style\IndexKey{linestyle} of the border, possible values are \texttt{solid},\texttt{dashed},
% \texttt{underlined}, \texttt{beveled},and \texttt{inset}. Links do not support
% the \texttt{beveled} and \texttt{inset} options.
%    \begin{macrocode}
\define@choicekey{eforms}{linestyle}[\val\nr]{solid,dashed,underlined,%
    beveled,inset}
{%
    \ifcase\nr\relax
        \@eqS{S}\or\@eqS{D}\or\@eqS{U}%
        \or\@eqS{B}\or\@eqS{I}\else\@eqS{S}%
    \fi
}
%    \end{macrocode}
%
% When a line style of dashed is chosen, you can specify a dashed array\IndexKey{dasharray}.
% The default is 3, which means 3 points of line, 3 points gap.
% A value of \texttt{dashedarray=3 2} means three points of line, followed
% by two points of space.
%    \begin{macrocode}
\define@key{eforms}{dasharray}[3]{\@eqD{#1}}
%    \end{macrocode}
% Set the color of the link text. Ignored if the colorlinks option of hyperref
% has not been taken.  The value of \texttt{linktxtcolor}\IndexKey{linktxtcolor} is a named color. For example,
% \texttt{linkcolor=red}. The default is |\@linkcolor| from hyperref. This default
% can be changed by redefining |\@linkcolor|, or be redefining |\defaultlinkcolor|.
% If |linktxtcolor={}| (an empty argument), or simply \texttt{linktxtcolor}, no color is applied to the text.
%    \begin{macrocode}
\define@key{eforms}{linktxtcolor}[]{%
    \let\ef@linktxtcolor@set=1\@eqlinktxtcolor{#1}}
\let\ef@linktxtcolor@set=0
%    \end{macrocode}
%
% \subsection{The General and Option Tab}
% In the general and option tabs of Acrobat a variety of attributes
% of the field can be set, such visibility, printability, readonly,
% orientation, captions (for button), values, default values, etc.
% can be set.
%
% The \texttt{annotflags}\IndexKey{annotflags} is a bit field, possible values are \texttt{hidden}, \texttt{print},
% \texttt{-print}, \texttt{noview}, and \texttt{lock}. Multiple values can
% be specified. The values are ``or-ed'' together. Most all forms are printable
% by default. If you don't want a form field to print specify \texttt{-print}.
% For example, |annotflags={-print,lock}| makes the field not printable and is
% locked, so the field cannot be moved through the UI.
%    \begin{macrocode}
\define@key{eforms}{annotflags}[]{\ef@flagtoks={}%
    \setkeys{annotflags}{#1}}
\@tfor\ef@flagopts:={{hidden}{FHidden}}{{print}{FPrint}}{{noprint}%
    {FNoPrint}}{{-print}{FNoPrint}}{{noview}{FNoView}}{{lock}{FLock}}%
    \do{\expandafter\getargs\ef@flagopts
    \edef\temp@expand@def{%
        \noexpand\define@key{annotflags}{\aeb@argi}[true]{%
            \noexpand\ef@flagtoks=%
                \noexpand\expandafter{\noexpand\ef@passedArgs}%
            \noexpand\edef\noexpand\ef@passedArgs{\noexpandiii%
                \F{\noexpandiii\csname\aeb@argii\noexpandiii\endcsname}%
                \noexpand\the\noexpand\ef@flagtoks}%
        }%
    }\temp@expand@def
}
%    \end{macrocode}
% A large number of fields that sets a number of properties of a field. Some of these
% are specified through the bitfield \texttt{fieldflags}\IndexKey{fieldflags}.
% These appear in the first of the paired groups listed below. Normally, a document
% author would not specify \texttt{radio}, \texttt{pushbutton} or \texttt{combo}.
% These properties are used
% by eforms to construct a radio button field, a push button and a combo box.
% The others can be used as appropriate.
%    \begin{macrocode}
\define@key{eforms}{fieldflags}[]{\ef@flagtoks={}%
    \setkeys{fieldflags}{#1}}
\@tfor\ef@flagopts:={{readonly}{FfReadOnly}}{{required}{FfRequired}}%
    {{noexport}{FfNoExport}}{{multiline}{FfMultiline}}%
    {{password}{FfPassword}}{{notoggleoff}{FfNoToggleToOff}}%
    {{radio}{FfRadio}}{{pushbutton}{FfPushButton}}{{combo}{FfCombo}}%
    {{edit}{FfEdit}}{{sort}{FfSort}}{{fileselect}{FfFileSelect}}%
    {{multiselect}{FfMultiSelect}}{{nospellcheck}{FfDoNotSpellCheck}}%
    {{noscrolling}{FfDoNotScroll}}{{comb}{FfComb}}%
    {{radiosinunison}{FfRadiosInUnison}}%
    {{commitonchange}{FfCommitOnSelChange}}{{richtext}{FfRichText}}\do{%
    \expandafter\getargs\ef@flagopts
    \edef\temp@expand@def{%
        \noexpand\define@key{fieldflags}{\aeb@argi}[true]%
        {%
            \noexpand\ef@flagtoks=\noexpand\expandafter{%
                \noexpand\ef@passedArgs}%
            \noexpand\edef\noexpand\ef@passedArgs{\noexpandiii%
               \Ff{\noexpandiii\csname\aeb@argii\noexpandiii\endcsname}%
               \noexpand\the\noexpand\ef@flagtoks}%
        }%
    }\temp@expand@def
}
%    \end{macrocode}
% Enter a text value to appear as a tool tip\IndexKey{tooltip}. For example,
% \texttt{tooltip={Hi, press me and see what happens!}}. Enclose in
% parentheses if the text string contains a comma.
%    \begin{macrocode}
\define@key{eforms}{tooltip}{\@eqTU{#1}}
%    \end{macrocode}
% The default value\IndexKey{default}\IndexKey{defaultstyle} of a field (text, list, combobox) what is used
% when the field is reset. Example: \texttt{default=Name}
%    \begin{macrocode}
\define@key{eforms}{default}{\@eqDV{#1}}
\define@key{eforms}{defaultstyle}{\@eqDS{#1}}
%    \end{macrocode}
% The \texttt{value}\IndexKey{value}\IndexKey{richvalue}\IndexKey{apprD} of the field (text, list, combobox). Example: \texttt{value=AcroTeX}.
%    \begin{macrocode}
\define@key{eforms}{value}{\@eqV{#1}}
\define@key{eforms}{richvalue}{\@eqRV{#1}}
\define@key{eforms}{apprD}{\@eqAP{#1}}
%    \end{macrocode}
% Set the orientation\IndexKey{rotate} of the field, values are 0, 90, 180 and 270. If 90 or 270
% are chosen, the height and width of the field need to be reversed. This is not
% done automatically by eForms (probably should).
%    \begin{macrocode}
\define@choicekey{eforms}{rotate}[\val\nr]{0,90,180,270}
    {\expandafter\@eqR\expandafter{\val}}{}
%    \end{macrocode}
% The background color\IndexKey{bgcolor} of the field. This is an RGB value.
%    \begin{macrocode}
\define@key{eforms}{bgcolor}[]{\@eqBG{#1}}
%    \end{macrocode}
% The normal text\IndexKey{uptxt} that appears on a button. Example \texttt{uptxt=Push Me}
%    \begin{macrocode}
\define@key{eforms}{uptxt}{\@eqCA{#1}}
%    \end{macrocode}
% The (mouse) down text\IndexKey{downtxt} of the button caption. Example: \texttt{downtxt=Thanks!}
%    \begin{macrocode}
\define@key{eforms}{downtxt}{\@eqAC{#1}}
%    \end{macrocode}
% The (mouse) rollover\IndexKey{rollovertxt} caption of a button. Example: \texttt{rollovertxt=AcroTeX!}
%    \begin{macrocode}
\define@key{eforms}{rollovertxt}{\@eqRC{#1}}
%    \end{macrocode}
% The type of alignment\IndexKey{align} of a text field. Permitted values are
% \texttt{left}, \texttt{centered}, and \texttt{right}.
%    \begin{macrocode}
\define@choicekey{eforms}{align}[\val\nr]{left,centered,right}{%
    \ifx\annot@type\annot@type@text
        \expandafter\@eqQ\expandafter{\nr}\fi}{}
%    \end{macrocode}
% The text font\IndexKey{textfont} to be used with the text of the field.
%    \begin{macrocode}
\define@key{eforms}{textfont}{\@eqtextFont{#1}}
%    \end{macrocode}
% The text size\IndexKey{textsize} to be used. A value of 0 means auto size.
%    \begin{macrocode}
\define@key{eforms}{textsize}{\@eqtextSize{#1}}
%    \end{macrocode}
% The color of the text\IndexKey{textcolor} in the field. This can be in \textsf{G}, \textsf{RGB} or \textsf{CMYK} color
% space.
%    \begin{macrocode}
\define@key{eforms}{textcolor}{%
    \@eqtextColor{#1}%
%    \aeb@cntargs#1 \\\relax
%    \ifcase\aeb@nArgs\relax\or
%    \@eqtextColor{#1 g}\or\or
%    \@eqtextColor{#1 rg}\or
%    \@eqtextColor{#1 k}\fi
}
%    \end{macrocode}
% Use \texttt{maxlength}\IndexKey{maxlength} to limit the number of characters input into a text field.
% Example: \texttt{maxlength=12}.
%    \begin{macrocode}
\define@key{eforms}{maxlength}{\@eqMaxLen{#1}}
%    \end{macrocode}
% \paragraph*{Icon Appearances} We begin a section for creating icon
% appearances for a push button. We define three keys \texttt{normappr}\IndexKey{normappr},
% \texttt{rollappr}\IndexKey{rollappr}, and \texttt{downappr}\IndexKey{downappr}
%    \begin{macrocode}
\define@key{eforms}{normappr}{\@eqI{#1}}
\define@key{eforms}{rollappr}{\@eqRI{#1}}
\define@key{eforms}{downappr}{\@eqIX{#1}}
\define@choicekey{eforms}{layout}[\val\nr]{labelonly,icononly,%
    icontop,iconbottom,iconleft,iconright,labelover}{%
    \ifx\annot@type\annot@type@button
        \expandafter\@eqTP\expandafter{\nr}\fi}{}
%    \end{macrocode}
% Scaling keys, \texttt{scalewhen}\IndexKey{scalewhen} and \texttt{scale}\IndexKey{scale}
%    \begin{macrocode}
\define@choicekey{eforms}{scalewhen}[\val\nr]{always,never,%
    iconbig,iconsmall}{%
    \ifx\annot@type\annot@type@button
    \ifcase\nr\relax
        \def\eq@@SW{A}\or
        \def\eq@@SW{N}\or
        \def\eq@@SW{B}\or
        \def\eq@@SW{S}\else
        \def\eq@@SW{A}\fi
    \expandafter\@eqSW\expandafter{\eq@@SW}\fi}{}
\define@choicekey{eforms}{scale}[\val\nr]{proportional,%
    nonproportional}{%
    \ifx\annot@type\annot@type@button
    \ifcase\nr\relax
        \def\eq@@ST{P}\or
        \def\eq@@ST{A}\else
        \def\eq@@ST{P}\fi
    \expandafter\@eqST\expandafter{\eq@@ST}\fi}{}
%    \end{macrocode}
% Positioning keys, \texttt{position}\IndexKey{position} and \texttt{fitbounds}\IndexKey{fitbounds}
%    \begin{macrocode}
\define@key{eforms}{position}{\@eqPA{#1}}
\define@choicekey{eforms}{fitbounds}[\val\nr]{true,false}[true]{%
    \ifx\annot@type\annot@type@button
    \ifcase\nr\relax
        \def\eq@@FB{true}\or
        \def\eq@@FB{false}\else
        \def\eq@@FB{false}\fi
    \expandafter\@eqFB\expandafter{\eq@@FB}\fi}{}
%    \end{macrocode}
% This set of key-values\IndexKey{norm}\IndexKey{roll}\IndexKey{down} are designed to fill in the AP dictionary for the check box
% and the radio button field.
%\begin{verbatim}
% appr={norm={on={},off={}}},roll={on={},off={}}},down={on={},off={}}}}
% \AP{/N << /On {xO} /Off {xX} >>  /R << /On {xX} /Off {xO} >>}
%\end{verbatim}
%    \begin{macrocode}
\define@key{efappr}{norm}[]{\def\efappr@norm{#1}}
\define@key{efappr}{roll}[]{\def\efappr@roll{#1}}
\define@key{efappr}{down}[]{\def\efappr@down{#1}}
\setkeys{efappr}{norm,roll,down}
\define@key{efappr@state}{on}[]{%
    \expandafter\def\csname \ef@state @on\endcsname{#1}}
\define@key{efappr@state}{off}[]{%
    \expandafter\def\csname \ef@state @off\endcsname{#1}}
\let\norm@on\@empty\let\norm@off\@empty
\let\roll@on\@empty\let\roll@off\@empty
\let\down@on\@empty\let\down@off\@empty
%    \end{macrocode}
% The key \texttt{appr}\IndexKey{appr} corresponds to the \textbf{AP} dictionary.
% and the radio button field.
%\begin{macrocode}
\define@key{eforms}{appr}{\setkeys{efappr}{#1}%
    \def\ef@state{norm}%
    \edef\ef@edef@tmp{\noexpand\setkeys{efappr@state}{\efappr@norm}}%
    \ef@edef@tmp\def\ef@state{roll}%
    \edef\ef@edef@tmp{\noexpand\setkeys{efappr@state}{\efappr@roll}}%
    \ef@edef@tmp\def\ef@state{down}%
    \edef\ef@edef@tmp{\noexpand\setkeys{efappr@state}{\efappr@down}}%
    \ef@edef@tmp\def\eq@AP{%
        /AP<<
            \ifx\efappr@norm\@empty\else/N <<
                \eq@On\space{\norm@on}/Off {\norm@off}>>\fi
            \ifx\efappr@roll\@empty\else/R <<
                \eq@On\space{\roll@on}/Off {\roll@off}>>\fi
            \ifx\efappr@down\@empty\else/D <<
                \eq@On\space{\down@on}/Off {\down@off}>>\fi
        \space>>
    }%
}
%    \end{macrocode}%
% The \texttt{autocenter}\IndexKey{autocenter} is a feature of eforms. Use \texttt{autocenter=yes} (the default) to center the bounding
% box, and use \texttt{autocenter=no} otherwise.
%    \begin{macrocode}
\define@choicekey{eforms}{autocenter}[\val\nr]{yes,no}
{%
    \ifcase\nr\relax\@eqautoCenter{y}\or
    \@eqautoCenter{n}\fi
}{}
\define@choicekey{eforms}{inline}[\val\nr]{yes,no}
{%
    \ifcase\nr\relax\@eqinline{y}\or
    \@eqinline{n}\fi
}{}
%    \end{macrocode}
% Set presets\IndexKey{presets} from inside a \cs{ui} argument. For example,
%\begin{verbatim}
%\def\myUIOptsi{%
%    border=visible,         % Link Type
%    linktxtcolor=blue,      % blue link color
%    linewidth=medium,       % Line thickness
%    highlight=outline,      % Highlight style
%    linestyle=dashed,       % Line style
%    bordercolor={1 0 0},    % border color
%    js={app.alert("Good, good, good!")},
%}
%\end{verbatim}
% Then we can say,
%\begin{verbatim}
%\setLinkText[\ui{presets={\myUIOptsii}}]{Press Me Again!!}
%\end{verbatim}
%    \begin{macrocode}
\define@key{eforms}{presets}{%
    \ef@jstoks=\expandafter{#1}%
    \edef\ef@temp@expand{\noexpand\setkeys{eforms}{\the\ef@jstoks}}%
    \ef@temp@expand
}
%    \end{macrocode}
% \texttt{symbolchoice}\IndexKey{symbolchoice}is used with a checkbox or radio button field. This sets the symbol
% that appears in the field with the box is checked, choices are
% \texttt{check}, \texttt{circle}, \texttt{cross}, \texttt{diamond},
% \texttt{square}, and \texttt{star}.
%    \begin{macrocode}
\define@choicekey{eforms}{symbolchoice}[\val\nr]%
    {check,circle,cross,diamond,square,star}
    {\expandafter\@eqsymbolchoice\expandafter{\val}}{}
%    \end{macrocode}
%
% \subsection{The Signed Tab}
%
% A signature field has a signed tab. On that tab is an option to mark a set of fields
% as readonly (locked). The locked key controls that option.
%
% The \texttt{lock}\IndexKey{lock} key is used with signature fields, currently, there is
% no nice user interface to this key. Typical entries are
%\begin{verbatim}
%   lock={/Action/All}          % lock all fields in the doc
%   lock={/Action/Include       % lock all fields listed in Fields
%         /Fields [(field1)(field2)...]}
%   lock={/Action/Exclude       % lock  all fields not listed in Fields
%         /Fields [(field1)(field2)...]}
%\end{verbatim}
%
%    \begin{macrocode}
\define@key{eforms}{lock}{\@eqLock{#1}}
%    \end{macrocode}
%
% Another option that is included in the Signed tab is titled ``This script executes
% when field is signed.''
%
% This is an option that, through the user interface, is mutually exclusive from
% locking fields. This option appears to be implemented through the format event.
% Thus, to populate this option with JavaScript use \texttt{format}. For example,
%\begin{verbatim}
%   format={app.alert("Thank you for signing this field.");}
%\end{verbatim}
%
% \subsection{Specifying Actions}
% Here, we offer up various common actions for a document author to choose from.
%
% The \texttt{goto}\IndexKey{goto} key is a combines \textsf{GoTo} and \textsf{GoToR} action. There are options for jumping to a page
% with a particular view, a latex label, or a named destination.
%
% The goto key first calls |\setkeys{efgoto}{#1}|, which parses the parameters. Following the
% reading of the parameters, we determined whether this is a GoTo or a GoToR request, whether
% we are jumping to a page, a named destination or a latex label.
%
% Documentation for the parameters of \texttt{goto} follow this definition.
%
% Process the \texttt{goto} key using \pkg{conv-xkv}.
% \changes{v2.9g}{2017/01/03}{Process the \string\texttt{goto} key using \string\pkg{conv-xkv}}
%    \begin{macrocode}
\define@key{eforms}{goto}[]{%
%    \end{macrocode}
%    (2017/01/03) Use \cs{cxkvsetkeys} for \texttt{goto}
%    \changes{v2.9g}{2017/01/03}{Use \string\cs{cxkvsetkeys} for \string\texttt{goto}}
%    \begin{macrocode}
    \cxkvsetkeys{efgoto}{#1}%
    \ifx\ef@goto@url\@empty
        \ifx\ef@goto@file\@empty
            % Jump within the file
            \def\ef@subtype{/S/GoTo }%
            \ifcase\eq@drivernum
                \def\ef@formatpage{{Page\ef@page}}%
            \or
                \def\ef@formatpage{\pdfpageref\ef@page\space\space 0 R}%
            \or
                \def\ef@formatpage{@page\ef@page}%
            \fi
            \let\ef@open\@empty
            \let\ef@formatfile\@empty
        \else
            % Jump to another PDF
%            \ifx\ef@linktxtcolor@set0%
%                \def\ef@thislinkcolor{\@filecolor}\fi
            \def\ef@subtype{/S/GoToR }%
            \count0=\ef@page\advance\count0by-1
            \edef\ef@formatpage{\the\count0 }%
            \def\ef@formatfile{/F (\ef@goto@file)}%
        \fi
        \ifx\ef@goto@targetdest\@empty
            \ifx\ef@labeldest\@empty
            % we will jump to a page, it might be the default page
                \expandafter\@eqA\expandafter{\ef@subtype
                    /D[\ef@formatpage\ef@view]%
                    \ef@formatfile\ef@open}%
            \else
            % jump to a label
                \expandafter\@eqA\expandafter{\ef@subtype%
                    /D (\labelRef{\ef@labeldest})%
                    \ef@formatfile\ef@open}%
            \fi
        \else
        % jump to a target
            \expandafter\@eqA\expandafter{\ef@subtype%
                /D (\ef@goto@targetdest)%
                \ef@formatfile\ef@open}%
        \fi
    \else % go to url
        \ifx\ef@goto@openparams\@empty
            \@eqA{/S/URI/URI(\ef@goto@url)}\else
            \@eqA{/S/URI/URI(\ef@goto@url\#\ef@goto@openparams)}%
        \fi
    \fi
}
%    \end{macrocode}
% The keys for \texttt{goto} are \texttt{file},
% \texttt{targetdest}, \texttt{labeldest}, \texttt{page},
% \texttt{view}, and \texttt{open}.
% Specify a relative path\IndexKey{file} to the PDF file. This will work on the Web if
% the position is the same relative to the calling file.
%    \begin{macrocode}
\define@key{efgoto}{file}[]{\def\ef@goto@file{#1}}
\let\ef@goto@file\@empty
%    \end{macrocode}
% Specify a url\IndexKey{url} to create a weblink
%    \begin{macrocode}
\define@key{efgoto}{url}[]{%
    \if\ef@linktxtcolor@set0\@eqlinktxtcolor{\@urlcolor}\fi
    \def\ef@goto@url{#1}%
}
\let\ef@goto@url\@empty
%    \end{macrocode}
% Specify a relative path to the PDF file with \texttt{openparams}\IndexKey{openparams}.
% This will work on the Web if the position is the same relative to the calling file.
%    \begin{macrocode}
\define@key{efgoto}{openparams}[]{\def\ef@goto@openparams{#1}}
\let\ef@goto@openparams\@empty
%    \end{macrocode}
%    Jump to a target\IndexKey{targetdest}, perhaps created by |\hypertarget|. For example,
%    if we say |\hypertarget{acrotex}{Welcome!}|, we can then jump to
%    \texttt{acrotex} by specifying \texttt{targetdest=acrotex}.
%    \begin{macrocode}
\define@key{efgoto}{targetdest}[]{\def\ef@goto@targetdest{#1}}
\let\ef@goto@targetdest\@empty
%    \end{macrocode}
%    \texttt{labeldest}\IndexKey{labeldest} is the
%    same as targetdest, but now we jump to a destination specified by
%    a latex label. For example, |\section{AcroTeX}\label{acrotex}|,
%    we can jump to this section by specifying \texttt{labeldest=acrotex}.
%    \begin{macrocode}
\define@key{efgoto}{labeldest}[]{\def\ef@labeldest{#1}}
\let\ef@labeldest\@empty
%    \end{macrocode}
% The page number\IndexKey{page} that you want to jump to. If we set \texttt{page=1},
% we will jump to the first page of the document.
%    \begin{macrocode}
\define@key{efgoto}{page}[1]{\def\ef@page{#1}}
\def\ef@page{1}
\def\ef@view{/Fit}%
%    \end{macrocode}
% The view\IndexKey{view} can be set when the page key is used. Possible values are
% \texttt{fitpage}, \texttt{actualsize}, \texttt{fitwidth},
% \texttt{fitvisible}, and \texttt{inheritzoom}. These terms correspond
% to Acrobat's UI. When jumping to a destination, the view is set by the
% destination code.
%\begin{verbatim}
%\def\dl@fitpage{/Fit}
%\def\dl@actualsize{/XYZ -32768 -32768 1.0}
%\def\dl@fitwidth{/FitH -32768}
%\def\dl@fitvisible{/FitBH -32768}
%\def\dl@inheritzoom{/XYZ 0 0 0}
%\end{verbatim}
%    \begin{macrocode}
\define@choicekey{efgoto}{view}[\val\nr]{fitpage,actualsize,fitwidth,%
    fitheight,fitvisible,inheritzoom}
{%
    \edef\ef@view{\csname dl@\val\endcsname}%
}{}
%    \end{macrocode}
% The \texttt{open}\IndexKey{open} key is use when you specify the \texttt{file} key. The
% open key determines if a new window is opened or not when we
% jump to the file. Possible values are \texttt{userpref} (use user preferences),
% \texttt{new} (open new window), \texttt{existing} (use the existing window).
%    \begin{macrocode}
\define@choicekey{efgoto}{open}[\val\nr]{userpref,new,existing}
{%
    \ifcase\nr\relax
    \let\ef@open\@empty\or
    \def\ef@open{/NewWindow true }\or
    \def\ef@open{/NewWindow false }\fi
}{}
\let\ef@open\@empty
%    \end{macrocode}
% Through the \texttt{launch}\IndexKey{launch} key, we support the subtype \textbf{Launch}. The launch action dictionary
% contains six keys, four of which we support: \texttt{S}, \texttt{F}, \texttt{Win},
% and \texttt{New Window} (boolean). The Win key is a dictionary with keys \texttt{F}, \texttt{D}, \texttt{O},
% and \texttt{P}. If the \texttt{Win} key is used, the F key in the launch dictionary is optional. The parameters
% of the Win dictionary are passed to the \textsf{ShellExecute}.
%
% Process the \texttt{launch} key using \pkg{conv-xkv}.
% \changes{v2.9g}{2017/01/03}{Process the \string\texttt{launch} key using \string\pkg{conv-xkv}}
%    \begin{macrocode}
\define@key{eforms}{launch}[]{%
%    \end{macrocode}
%    (2017/01/03) Use \cs{cxkvsetkeys} for \texttt{launch}
%    \changes{v2.9g}{2017/01/03}{Use \string\cs{cxkvsetkeys} for \string\texttt{launch}}
%    \begin{macrocode}
    \cxkvsetkeys{eflaunch}{#1}%
    \@eqA{/S/Launch\ifx\ef@launch@file\@empty\else
        /F(\ef@launch@file)\fi\ef@launch@open
        \ifx\ef@launch@win\@empty\else
            /Win<<\ifx\ef@launchwin@file\@empty
            /F(\ef@launch@file)\else/F(\ef@launchwin@file)\fi
            \ifx\ef@launchwin@params\@empty\else
                /P(\ef@launchwin@params)\fi
            \ifx\ef@launchwin@open\@empty\else
                /O(\ef@launchwin@open)\fi
            \ifx\ef@launchwin@dir\@empty\else
                /D(\ef@launchwin@dir)\fi>>
        \fi
        }%
}
%    \end{macrocode}
%    The value of the \texttt{file} key can be an application or a file. If a
%    document, the operating system will launch an application associated with
%    the file extension. The path can be given relative to the current source
%    folder.
%    \begin{macrocode}
\define@key{eflaunch}{file}[]{\def\ef@launch@file{#1}}
\let\ef@launch@file\@empty
%    \end{macrocode}
%    The value of the \texttt{open} key is ignored, if the value of the
%    \texttt{file} key is not a PDF file.  If the file is a PDF, a new window
%    can be optionally opened, or not. The default is to use the user
%    preferences.
%    \begin{macrocode}
\define@choicekey{eflaunch}{open}[\val\nr]{userpref,new,existing}
{%
    \ifcase\nr\relax
    \let\ef@launch@open\@empty\or
    \def\ef@launch@open{/NewWindow true }\or
    \def\ef@launch@open{/NewWindow false }\fi
}{}
\let\ef@launch@open\@empty
%    \end{macrocode}
%    The PDF Specification allows for additional parameters to be passed on
%    the windows operating system. (No such key is available for the Mac or
%    for Unix.) This is a windows-only feature. The \texttt{winParams} key
%    itself takes key values pairs; these keys are \texttt{file} (\texttt{F}),
%    \texttt{directory} (\texttt{D}), \texttt{open} (\texttt{O}), and
%    \texttt{params} (\texttt{P}), these keys are defined below.
%    \begin{macrocode}
\define@key{eflaunch}{winParams}[]{\def\ef@launch@win{#1}%
    \setkeys{eflaunchwin}{#1}%
}\let\ef@launch@win\@empty
%    \end{macrocode}
%    As far as I can see, the \texttt{file} (application or file) must have a
%    full path (absolute path). The path should be enclosed in double quotes if
%    the path contains any spaces.
%    \begin{macrocode}
\define@key{eflaunchwin}{file}[]{\def\ef@launchwin@file{#1}}
\let\ef@launchwin@file\@empty
%    \end{macrocode}
% As far as I can see, this key does nothing. The value of the key
% is a path to the (startup) folder.
%    \begin{macrocode}
\define@key{eflaunchwin}{directory}[]{\def\ef@launchwin@dir{#1}}
\let\ef@launchwin@dir\@empty
%    \end{macrocode}
%    The \texttt{open} key takes one of two (documented) values: \texttt{open}
%    (the default) or \texttt{print}. But because these parameters are passed
%    to Window's \texttt{ShellExecute} a value of explore is recognized as well
%    (when the value of \texttt{file} is a path to a folder).
%    \begin{macrocode}
\define@key{eflaunchwin}{open}[]{\def\ef@launchwin@open{#1}}
\let\ef@launchwin@open\@empty
%    \end{macrocode}
% The launch parameters. Any paths must be absolute and enclosed in double quotes, if the
% path contains a space.
%    \begin{macrocode}
\define@key{eflaunchwin}{params}[]{\def\ef@launchwin@params{#1}}
\let\ef@launchwin@params\@empty
%    \end{macrocode}
% The \texttt{js}\IndexKey{js} key is used to execute JavaScript actions on a mouse up trigger.
% The argument is a JavaScript text string: |js={app.alert("Hello World!"}|.
% The value of \texttt{js} may be a macro containing JavaScript, which would include
% a macro created by the \texttt{defineJS} environment of \textsf{insdljs}.
%    \begin{macrocode}
\define@key{eforms}{js}[]{\@eqA{\JS{#1}}}
%    \end{macrocode}
% Next up are additional actions, and there are a lot of them. All these
% take JavaScript code as their values.%
% \begin{itemize}
%   \item \texttt{mouseup}\IndexKey{mouseup}: Executes its code with a mouse up event. If there is a JavaScript
%         action defined by the \texttt{js} key (or the |\A| key), the \texttt{js} (|\A|) action is executed.
%   \item \texttt{mousedown}\IndexKey{mousedown}: Executes when the mouse is hovering over the field and the user clicks
%         on the mouse.
%   \item \texttt{onenter}\IndexKey{onenter}: Executes its code when the user moves the mouse into the form field (the bounding rectangle).
%   \item \texttt{onexit}\IndexKey{onexit}: Executes its code when the user moves the mouse out of the form field (the bounding rectangle).
%   \item \texttt{onfocus}\IndexKey{onfocus}: Executes its code when the user brings the field into focus.
%   \item \texttt{onblur}\IndexKey{onblur}: Executes its code when the user brings the field loses focus (the user tabs away from
%         the field, or click outside the field).
%   \item \texttt{format}\IndexKey{format}: JavaScript to format the text that appears to
%           the user in a text field or editable combo box.
%   \item \texttt{keystroke}\IndexKey{keystroke}: JavaScript to process each keystroke in a text field or editable combo box.
%   \item \texttt{validate}\IndexKey{validate}: JavaScript to validate the committed data input into a text field or editable combo box.
%   \item \texttt{calculate}\IndexKey{calculate}: JavaScript to make calculates based on the values of other fields.
%   \item \texttt{pageopen}\IndexKey{pageopen}: JavaScript that executes when the page containing the field is opened.
%   \item \texttt{pageclose}\IndexKey{pageclose}: JavaScript that executes when the page containing the field is closed.
%   \item \texttt{pagevisible}\IndexKey{pagevisible}: JavaScript that executes when the page containing the field first becomes visible to the user.
%   \item \texttt{pageinvisible}\IndexKey{pageinvisible}: JavaScript that executes when the page containing the field is no longer visible to the user.
% \end{itemize}
%    \begin{macrocode}
\@tfor\ef@AActions:={{mouseup}{AAmouseup}}{{mousedown}{AAmousedown}}%
    {{onenter}{AAmouseenter}}{{onexit}{AAmouseexit}}%
    {{onfocus}{AAonfocus}}{{onblur}{AAonblur}}%
    {{format}{AAformat}}{{keystroke}{AAkeystroke}}%
    {{validate}{AAvalidate}}{{calculate}{AAcalculate}}%
    {{pageopen}{AApageopen}}{{pageclose}{AApageclose}}%
    {{pagevisible}{AApagevisible}}%
    {{pageinvisible}{AApageinvisible}}\do{%
    \expandafter\getargs\ef@AActions\ef@jstoks={#1}%
    \edef\temp@expand@def{\noexpand\define@key{eforms}{\aeb@argi}[]%
    {\noexpand\csname @eq\aeb@argii\noexpand\endcsname%
        {\the\ef@jstoks}}}%
    \temp@expand@def
}
%    \end{macrocode}
% The \texttt{objdef}\IndexKey{objdef} key is used to create indirect references to the form field.
% The value of this key must be unique throughout the whole document. Both \texttt{objdef}
% and \texttt{taborder}\IndexKey{taborder} are used for structured tabbing. (Distiller only)
%    \begin{macrocode}
\define@key{eforms}{objdef}{\@eqobjdef{#1}}
\define@key{eforms}{taborder}{\@eqtaborder{#1}}
%    \end{macrocode}
%    \begin{macrocode}
%</userinterface>
%    \end{macrocode}
%
%    \section{Input Driver Specific Code}
% Now bring in driver dependent macros.  The macros defined are
% the answer macros for the \texttt{shortquiz} environment and most all
% for the \texttt{quiz} environment.  These macros make heavy use of
% JavaScript.  The method of introducing these JavaScripts and
% PDF code related elements depends on the driver.
% \changes{v2.9j}{2017/01/22}{\string\cs{eq@rectH} and \string\cs{eq@rectW} pass through \string\cs{setlength}
% to enable arithmetic for the arguments of those commands}
%    \begin{macrocode}
%<*package>
%    \end{macrocode}
%    \begin{macrocode}
\input{\eq@drivercode}
%    \end{macrocode}
%    \begin{macrocode}
%</package>
%    \end{macrocode}
%    \subsection{For the \texttt{dvips} and \texttt{dvipsone} options}
%    \begin{macrocode}
%<*epdfmark>
%    \end{macrocode}
% This is the code for the \texttt{dvipsone} and \texttt{dvips}
% options.  These two are done together.  \textsf{hyperref}
% redefines the macro \cmd{\literalps@out} appropriate to each of
% these drivers.  Both use pdfmarks, so we can put them together.
% \par\medskip\noindent
% This sets the rectangle size allowing for a literal
% hyperlink---meaning we can insert arbitrary links actions.
%    \begin{macrocode}
\def\Rect#1{\pdf@rect{\textcolor{\@linkcolor}{#1}}}
%    \end{macrocode}
% Code to hide the solutions page to a quiz that has solutions.
%    \begin{macrocode}
\def\noPeek#1#2{\literalps@out{%
     [ {ThisPage} << \noPeekAction{#1}{#2} >> /PUT pdfmark}}
%    \end{macrocode}
% Driver dependent code (distiller) for choice fields, list and combo.
%    \begin{macrocode}
\def\eq@choice@driver
{%
    \Hy@pdfstringtrue
    \if\@vertRotate1 \let\W@temp\eq@rectW\edef\eq@rectW{\eq@rectH}%
        \edef\eq@rectH{\W@temp}\fi\centerWidget\eq@rectH
    \ifx\autoCenter\ef@n\eqcenterWidget=0pt\fi
    \pdf@rect{\lower\eqcenterWidget\ef@Bbox{\eq@rectW}{\eq@rectH}}%
    \literalps@out{%
    [ \ifisCalculate/_objdef {\Fld@name}\else\eq@objdef\fi
      /Rect [pdf@llx pdf@lly pdf@urx pdf@ury]
        \common@choiceCode
    /ANN pdfmark
    \ifisCalculate[ {corder} {\Fld@name} /APPEND pdfmark\fi
    }\to@insertStrucTabOrder{Form}\endgroup
}
%    \end{macrocode}
% Driver dependent code (distiller) for push button fields.
%    \begin{macrocode}
\def\eq@Button@driver
{%
    \Hy@pdfstringtrue
    \ifx\eq@rectW\@empty\def\eq@rectW{\wd\eq@tmpbox}\ef@djXPD\fi
    \if\@vertRotate1 \let\W@temp\eq@rectW\edef\eq@rectW{\eq@rectH}%
        \edef\eq@rectH{\W@temp}\fi\centerWidget\eq@rectH
    \ifx\autoCenter\ef@n\eqcenterWidget=0pt\fi
    \pdf@rect{\lower\eqcenterWidget\ef@Bbox{\eq@rectW}{\eq@rectH}}%
    \literalps@out{%
    [ \eq@objdef/Rect [pdf@llx pdf@lly pdf@urx pdf@ury]
        \common@pushButtonCode
    /ANN pdfmark}\to@insertStrucTabOrder{Form}\endgroup
}
%    \end{macrocode}
% Driver dependent code (distiller) for radio and button fields.
%    \begin{macrocode}
\def\eq@RadioCheck@driver
{%
    \Hy@pdfstringtrue
    \if\@vertRotate1 \let\W@temp\eq@rectW\edef\eq@rectW{\eq@rectH}%
        \edef\eq@rectH{\W@temp}\fi\centerWidget\eq@rectH
    \ifx\autoCenter\ef@n\eqcenterWidget=0pt\fi
    \ifx\eq@rectW\@empty\def\eq@rectW{\wd\eq@tmpbox}\fi
    \pdf@rect{\lower\eqcenterWidget\ef@Bbox{\eq@rectW}{\eq@rectH}}%
    \literalps@out{%
    [ \eq@objdef/Rect [pdf@llx pdf@lly pdf@urx pdf@ury]
        \common@RadioCheckCode
    /ANN pdfmark}\to@insertStrucTabOrder{Form}\endgroup
}
\def\eq@l@check@driver
{%
    \pdf@rect{\makebox[\eq@tmpdima]{\phantom{\link@@Content}}}%
    \literalps@out{%
    [ \eq@objdef/Rect [pdf@llx pdf@lly pdf@urx pdf@ury]
        \common@RadioCheckCode
    /ANN pdfmark}\endgroup
}
%    \end{macrocode}
% Driver dependent code for text fields.
%    \begin{macrocode}
\def\eq@TextField{\Hy@pdfstringtrue
    \if\@vertRotate1 \let\W@temp\eq@rectW\edef\eq@rectW{\eq@rectH}%
        \edef\eq@rectH{\W@temp}\fi\centerWidget\eq@rectH
    \ifx\autoCenter\ef@n\eqcenterWidget=0pt\fi
    \pdf@rect{\lower\eqcenterWidget\ef@Bbox{\eq@rectW}{\eq@rectH}}%
    \literalps@out{%
    [ \ifisCalculate/_objdef {\Fld@name}\else\eq@objdef\fi
       /Rect [pdf@llx pdf@lly pdf@urx pdf@ury]
        \common@TextFieldCode
    /ANN pdfmark
    \ifisCalculate[ {corder} {\Fld@name} /APPEND pdfmark\fi
    }\to@insertStrucTabOrder{Form}\endgroup
}
%    \end{macrocode}
% \changes{v2.5p}{2012/09/25}{Corrected a bug in \string\cs{eq@SigField} for
% the dvipdfm-type drivers}
%    \begin{macrocode}
\def\eq@SigField{\Hy@pdfstringtrue
    \if\@vertRotate1 \let\W@temp\eq@rectW\edef\eq@rectW{\eq@rectH}%
        \edef\eq@rectH{\W@temp}\fi\centerWidget\eq@rectH
%    \centerWidget\eq@rectH
    \ifx\autoCenter\ef@n\eqcenterWidget=0pt\fi
    \pdf@rect{\lower\eqcenterWidget\ef@Bbox{\eq@rectW}{\eq@rectH}}%
    \literalps@out{%
    [ \eq@objdef/Rect [pdf@llx pdf@lly pdf@urx pdf@ury]
        \common@SigFieldCode
    /ANN pdfmark
    }\to@insertStrucTabOrder{Form}\endgroup
}
%    \end{macrocode}
% For processing the \texttt{pdfmark} with distiller, the key \texttt{/Action} is
% required (not \texttt{/A}). This macro converts \texttt{/A} to \texttt{/Action},
% and is used for the drivers using distiller.
%    \begin{macrocode}
\def\convertAToAction/A#1\@nil{\def\eq@A{/Action#1}}
%    \end{macrocode}
% Driver dependent code for links.
%    \begin{macrocode}
\def\setLink@driver
{%
    \ifx\eq@A\@empty\else\expandafter\convertAToAction\eq@A\@nil\fi
    \@eqBS{}%
    \pdf@rect{\link@@Box}%
    \literalps@out{%
    [ \eq@objdef/Rect [pdf@llx pdf@lly pdf@urx pdf@ury]
    \eq@Border
    \common@LinkCode
    /Subtype /Link
    /ANN pdfmark}%
    \to@insertStrucTabOrder{Link}\endgroup
}
\def\setLinkPbox@driver
{%
    \ifx\eq@A\@empty\else\expandafter\convertAToAction\eq@A\@nil\fi
    \@eqBS{}%
    \literalps@out{%
    [ \eq@objdef/Rect [\par@@Rect]
    \eq@Border
    \eq@QuadPoints  % QuadPoints
    \common@LinkCode
    /Subtype /Link
    /ANN pdfmark}%
    \to@insertStrucTabOrder{Link}\endgroup
}
%    \end{macrocode}
%    \begin{macrocode}
%</epdfmark>
%    \end{macrocode}
%    \subsection{For the \texttt{pdftex} option}
%    \begin{macrocode}
%<*epdftex>
%    \end{macrocode}
% Code used in the case of the \texttt{pdftex} option.
%\par\medskip\noindent
% Code to hide the solutions page to a quiz that has solutions.
%    \begin{macrocode}
\def\noPeek#1#2{\global\pdfpageattr=\expandafter{\noPeekAction{#1}{#2}}}
%    \end{macrocode}
% Support for automatic calculation fields for the \textsf{pdflatex} driver. The command
% \cs{HyField@AddToFields}, not modified here, is inserted at the end of the code
% for \cs{eq@choice@driver} and \cs{eq@TextField} below.
%\changes{v2.8}{2014/11/23}{Modified a macro from hyperref (2012/11/06) to support
%automatic calculation of fields using eforms (and hyperref).}
%    \begin{macrocode}
\def\HyField@@AddToFields#1{%
  \HyField@AfterAuxOpen{%
    \if@filesw
      \write\@mainaux{%
        \string\HyField@AuxAddToFields{#1}}%
% added for eforms
    \ifisCalculate\write\@mainaux{%
        \string\HyField@AuxAddToCoFields{}{#1}}\fi
% end eforms
    \fi
  }%
}%
%    \end{macrocode}
% driver dependent code for choice fields
%    \begin{macrocode}
\def\eq@choice@driver
{%
    \Hy@pdfstringtrue
    \if\@vertRotate1 \let\W@temp\eq@rectW\edef\eq@rectW{\eq@rectH}%
        \edef\eq@rectH{\W@temp}\fi\centerWidget\eq@rectH
    \ifx\autoCenter\ef@n\eqcenterWidget=0pt\fi
    \hbox{\pdfstartlink user{\common@choiceCode}%
    \lower\eqcenterWidget\ef@Bbox{\eq@rectW}{\eq@rectH}\pdfendlink}%
    \HyField@AddToFields
    \endgroup
}
\def\eq@Button@driver
{%
    \Hy@pdfstringtrue
    \ifx\eq@rectW\@empty\def\eq@rectW{\wd\eq@tmpbox}\ef@djXPD\fi
    \if\@vertRotate1 \let\W@temp\eq@rectW\edef\eq@rectW{\eq@rectH}%
        \edef\eq@rectH{\W@temp}\fi\centerWidget\eq@rectH
    \ifx\autoCenter\ef@n\eqcenterWidget=0pt\fi
    \ifx\eq@rectW\@empty\def\eq@rectW{\wd\eq@tmpbox}\fi
    \hbox{\pdfstartlink user{ \common@pushButtonCode }%
    \lower\eqcenterWidget\ef@Bbox{\eq@rectW}{\eq@rectH}\pdfendlink}%
    \endgroup
}
\def\eq@RadioCheck@driver
{%
    \Hy@pdfstringtrue
    \if\@vertRotate1 \let\W@temp\eq@rectW\edef\eq@rectW{\eq@rectH}%
        \edef\eq@rectH{\W@temp}\fi\centerWidget\eq@rectH
    \ifx\autoCenter\ef@n\eqcenterWidget=0pt\fi
    \ifx\eq@rectW\@empty\def\eq@rectW{\wd\eq@tmpbox}\fi
    \hbox{\pdfstartlink user{\common@RadioCheckCode}%
    \lower\eqcenterWidget\ef@Bbox{\eq@rectW}{\eq@rectH}\pdfendlink}%
    \endgroup
}
\def\eq@l@check@driver
{%
    \pdfstartlink user{\common@RadioCheckCode}%
    \makebox[\eq@tmpdima]{\phantom{\link@@Content}}%
    \pdfendlink\endgroup
}
\def\eq@TextField{\Hy@pdfstringtrue
    \if\@vertRotate1 \let\W@temp\eq@rectW\edef\eq@rectW{\eq@rectH}%
        \edef\eq@rectH{\W@temp}\fi\centerWidget\eq@rectH
    \ifx\autoCenter\ef@n\eqcenterWidget=0pt\fi
    \leavevmode
    \hbox{\pdfstartlink user{\common@TextFieldCode}%
    \lower\eqcenterWidget\ef@Bbox{\eq@rectW}{\eq@rectH}\pdfendlink}%
    \HyField@AddToFields
    \endgroup
}
\def\eq@SigField{\Hy@pdfstringtrue
    \if\@vertRotate1 \let\W@temp\eq@rectW\edef\eq@rectW{\eq@rectH}%
        \edef\eq@rectH{\W@temp}\fi\centerWidget\eq@rectH
    \ifx\autoCenter\ef@n\eqcenterWidget=0pt\fi
    \leavevmode\hbox{\pdfstartlink user{\common@SigFieldCode}%
    \lower\eqcenterWidget\ef@Bbox{\eq@rectW}{\eq@rectH}\pdfendlink}%
    \endgroup
}
\def\setLink@driver
{%
    \@eqBS{}%
    \leavevmode\pdfstartlink
    attr {\eq@Border}%
    user{/Subtype/Link \common@LinkCode}%
    \Hy@colorlink{\@linkcolor}\link@@Box
    \close@pdflink
    \endgroup
}
\def\ef@setTabOrder{\ifx\ef@taborder\@empty\else
    \edef\ef@tmp@toks{\the\pdfpageattr\space/Tabs/\ef@taborder}%
    \global\pdfpageattr=\expandafter{\ef@tmp@toks}%
    \fi\endgroup
}
%    \end{macrocode}
%    \begin{macrocode}
%</epdftex>
%    \end{macrocode}
%    \subsection{For the \texttt{dvipdfm}, \texttt{dvipdfmx}, \texttt{xetex} options}
%    \begin{macrocode}
%<*edvipdfm>
%    \end{macrocode}
% Code to hide the solutions page to a quiz that has solutions.
%    \begin{macrocode}
\def\noPeek#1#2{\@pdfm@mark{put @thispage << \noPeekAction{#1}{#2} >> }}
%    \end{macrocode}
%    (2016/12/22) Removed \cs{ef@adjHWxetex} in favor of \cs{ef@djXPD}.
%    \changes{v2.9d}{2016/12/22}{Removed \string\cs{ef@adjHWxetex} in favor of \string\cs{ef@djXPD}}
%    \begin{macrocode}
%\def\ef@adjHWxetex{%
%    \setlength{\@tempdima}{\eq@W@value bp}%
%    \setlength{\@tempdima}{2\@tempdima}%
%    \edef\ef@border@adj{\the\@tempdima}%
%    \setlength{\@tempdima}{\eq@rectH+\ef@border@adj}%
%    \edef\eq@rectH{\the\@tempdima}%
%    \setlength{\@tempdima}{\eq@rectW+\ef@border@adj}%
%    \edef\eq@rectW{\the\@tempdima}}
\let\ef@adjHWxetex\relax
%    \end{macrocode}
%    \begin{macrocode}
\def\eq@choice@driver{\ef@adjHWxetex
    \Hy@pdfstringtrue
    \if\@vertRotate1 \let\W@temp\eq@rectW\edef\eq@rectW{\eq@rectH}%
        \edef\eq@rectH{\W@temp}\fi\centerWidget\eq@rectH
    \ifx\autoCenter\ef@n\eqcenterWidget=0pt\fi
    \leavevmode\setbox\pdfm@box=%
        \hbox{\lower\eqcenterWidget\ef@Bbox{\eq@rectW}{\eq@rectH}}%
    \@pdfm@mark{ann @\Fld@name\space\dvipdfm@setdim
    <<\common@choiceCode>>}\unhbox\pdfm@box\relax%
    \@pdfm@mark{put @afields @\Fld@name}% record in @afields array
    \ifisCalculate\@pdfm@mark{put @corder @\Fld@name}\fi
\endgroup}
%    \end{macrocode}
% (2013/06/09) xelatex apparently includes the boundary in its width and height
% calculations. So we must too.
%    \begin{macrocode}
\def\eq@Button@driver{\Hy@pdfstringtrue
    \ifx\eq@rectW\@empty\def\eq@rectW{\wd\eq@tmpbox}\ef@djXPD\fi % 12/22
    \if\@vertRotate1 \let\W@temp\eq@rectW\edef\eq@rectW{\eq@rectH}%
        \edef\eq@rectH{\W@temp}\fi\centerWidget\eq@rectH
    \ifx\autoCenter\ef@n\eqcenterWidget=0pt\fi
    \setbox\pdfm@box=%
        \hbox{\lower\eqcenterWidget\ef@Bbox{\eq@rectW}{\eq@rectH}}%
    \@pdfm@mark{ann @\Fld@name\space\dvipdfm@setdim
    << \common@pushButtonCode >>}\unhbox\pdfm@box\relax%
    \@pdfm@mark{put @afields @\Fld@name}% record in @afields array
    \endgroup
}
\def\eq@RadioCheck@driver{\ef@adjHWxetex\Hy@pdfstringtrue
    \if\@vertRotate1 \let\W@temp\eq@rectW\edef\eq@rectW{\eq@rectH}%
        \edef\eq@rectH{\W@temp}\fi\centerWidget\eq@rectH
    \ifx\autoCenter\ef@n\eqcenterWidget=0pt\fi
    \ifx\eq@rectW\@empty\def\eq@rectW{\wd\eq@tmpbox}\fi
    \setbox\pdfm@box=%
        \hbox{\lower\eqcenterWidget\ef@Bbox{\eq@rectW}{\eq@rectH}}%
    \@pdfm@mark{ann \dvipdfm@setdim
    <<\common@RadioCheckCode>>}\unhbox\pdfm@box\relax%
    \endgroup
}
\def\eq@l@check@driver{%
    \setbox\pdfm@box=%
      \hbox{\makebox[\eq@tmpdima]{\phantom{\link@@Content}}}%
        \@pdfm@mark{ann \dvipdfm@setdim<<\common@RadioCheckCode>>}%
        \unhbox\pdfm@box\relax\endgroup
}
\def\eq@TextField{\ef@adjHWxetex\Hy@pdfstringtrue
    \if\@vertRotate1 \let\W@temp\eq@rectW\edef\eq@rectW{\eq@rectH}%
        \edef\eq@rectH{\W@temp}\fi\centerWidget\eq@rectH
    \ifx\autoCenter\ef@n\eqcenterWidget=0pt\fi
    \leavevmode\setbox\pdfm@box=%
        \hbox{\lower\eqcenterWidget\ef@Bbox{\eq@rectW}{\eq@rectH}}%
    \@pdfm@mark{ann @\Fld@name\space\dvipdfm@setdim
        << \common@TextFieldCode >>}\unhbox\pdfm@box\relax%
    \@pdfm@mark{put @afields @\Fld@name}% record in @afields array
    \ifisCalculate\@pdfm@mark{put @corder @\Fld@name}\fi
    \endgroup
}
\def\eq@SigField{\ef@adjHWxetex\Hy@pdfstringtrue
    \if\@vertRotate1 \let\W@temp\eq@rectW\edef\eq@rectW{\eq@rectH}%
        \edef\eq@rectH{\W@temp}\fi\centerWidget\eq@rectH
    \ifx\autoCenter\ef@n\eqcenterWidget=0pt\fi
    \leavevmode\setbox\pdfm@box=%
        \hbox{\lower\eqcenterWidget\ef@Bbox{\eq@rectW}{\eq@rectH}}%
    \@pdfm@mark{ann @\Fld@name\space\dvipdfm@setdim
        << \common@SigFieldCode >>}\unhbox\pdfm@box\relax%
    \endgroup
}
\def\setLink@driver{%
    \@eqBS{}\leavevmode
    \@pdfm@mark{bann
        <</Subtype/Link\eq@Border\common@LinkCode>>}%
        \Hy@colorlink{\@linkcolor}\link@@Box\Hy@endcolorlink
    \@pdfm@mark{eann}%
    \endgroup
}
\def\ef@setTabOrder{\ifx\ef@taborder\@empty\else
    \@pdfm@mark{ put @thispage << /Tabs/\ef@taborder >> }%
    \fi\endgroup
}
%</edvipdfm>
%    \end{macrocode}
%    \begin{macrocode}
%<*setcorder>
%    \end{macrocode}
%    \section{Document JavaScripts}
%    \subsection{Support for setting calculation order}
%    \begin{macrocode}
\begin{insDLJS}{cojs}{eforms: JavaScript to set calculation order}
var debugCalc=false;
ef_setCalcOrder.lastIndex=0;
function ef_setCalcOrder (a) {
    var o1, o2, f;
    while ( a.length > 0) {
        if (a.length > 1) {
            f=a.shift();
            o1=this.getField(f);
            if ( o1 == null ) {
                ef_CalcOrderErr(f);
                continue;
            }
            f = a[0];
            o2=this.getField(f);
            if ( o2 == null ) {
                ef_CalcOrderErr(f);
                a.shift();
                continue;
            }
            if (  o2.calcOrderIndex < o1.calcOrderIndex ) {
                o2.calcOrderIndex=o1.calcOrderIndex+1;
				ef_setCalcOrder.lastIndex=o2.calcOrderIndex;
			}
        } else {
            f=a.shift();
			o1=this.getField(f);
            if ( o1 == null ) {
                ef_CalcOrderErr(f);
                continue;
            }
			o1.calcOrderIndex=ef_setCalcOrder.lastIndex;			
		}
    }
}
function ef_CalcOrderErr(f) {
    console.show(); app.beep(0);
    console.println("calcOrder: the field \""+ f
        + "\" does not exist in this document, skipping it.\n\n"
        + "calcOrder: Check the case sensitive spelling of the field.");
}
var _EfCalcOrder=\efCalcOrder;
ef_setCalcOrder(_EfCalcOrder);
\end{insDLJS}
%</setcorder>
%    \end{macrocode}
%    \begin{macrocode}
%<*package>
\inputCalcOrderJS
%</package>
%    \end{macrocode}
%  \Finale
\endinput
